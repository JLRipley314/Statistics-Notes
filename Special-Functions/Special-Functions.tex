We briefly review the properties of some special functions that we use in the text.

%--------------------------------------------------------------------
\section{Exponential}
    The exponential function is $e^x$.
The Taylor series expansion of the exponential function about $x=0$ is
\begin{align}
    e^x
    =
    \sum_{n=0}^{\infty} \frac{1}{n!}x^n
    ,
\end{align}
this series has an infinite radius of convergence.
Another useful formula is
\begin{align}
    e^x
    =
    \lim_{n\to\infty}\left(1-\frac{x}{n}\right)^n
    .
\end{align}

%--------------------------------------------------------------------
\section{Gamma}
The gamma function is $\Gamma\left(x\right)$.
It is a continuous generalization of the factorial function $x!$.
The most common definitionof $\Gamma$ is
\begin{align}
    \label{eq:gamma-function-definition}
    \Gamma\left(x\right)
    =
    \int_0^{\infty}dt \; e^{-t} \; t^{x-1}   
    .
\end{align}
The Gamma function satisfies
\begin{align}
    \label{eq:gamma-recurence}
    \Gamma\left(x\right)
    =
    \left(x-1\right)\Gamma\left(x-1\right)
    .
\end{align}
We see that $\Gamma\left(1\right)=1$, and that $\Gamma\left(-n\right)=\infty$ for all $n=0,1,2,3,...$.
Notice that heuristically this ``had'' to be so, as from \eqref{eq:gamma-recurence}
\begin{align}
    \Gamma\left(1\right)
    =
    \left(1-1\right)\Gamma\left(0\right)
    =
    1
    .
\end{align}
For this to hold, we ``need'' $\Gamma\left(0\right)=\infty$.
