When it is hard or expensive to collect data, it can be useful to 
predict (or forecast) how well parameters of a given model could be measured
with simulated data.
Forecasting can inform whether a a more in-depth analysis of a model on real
data is worth doing--that is whether or not real data could place any meaningful
measurement of the parameters of a model.
Here we review several semi-analytic methods for forecasting. 

%===============================================================================
\section{Injection analysis}

We consider a model $P\left(\bx|\btheta\right)$ with prior $P\left(\btheta\right)$.
For whatever reason, we do not have any data $\bx$.
For example, it may be expensive to collect data, so we do not want to collect
it until we have some confidence that we could meaningfully measure parameters
in the model $P\left(\bx|\btheta\right)$.
An injection analysis involves determining the distribution of  
$P\left(\btheta_i|\bx_0\right)$ where
$\bx_0$ is fake (generated) data set that we hope represents a characteristic realization
of the data we may measure.
In other words, we have ``injected'' the data $\bx_0$ into our model.
If we can meaningfully measure/determine the parameters $\btheta$ given $\bx_0$,
it may be worth collecting real data/observations.

A reasonable choice for $\bx_0$ is to choose $\bx_0$ to maximize $P\left(\bx|\btheta_0\right)$,
where $\btheta_0$ are the values of parameters that you expect to hope to measure. 
In equations we choose $\bx_0$ to satisfy
\begin{align}
    \forall \bx \; P\left(\bx|\btheta_0\right) \leq P\left(\bx_0|\btheta_0\right)
\end{align}
Sometimes it is worth adding a realization of noise, $\bn$, to $\bx_0$; we call this
\begin{align}
    \bx_{0,n} = \bx_0 + \bn
\end{align}
For example, the components of $\bn$ may be drawn from a Gaussian with zero
mean and unit diagonal covariance matrix (although the choice of $\bn$ will depend
on your understanding nature of the experiment/observation).
It is common to $\bn=0$, which can be considered the ``best'' possible situation for
recovering parameters.
We then inject $\bx_{0,n}$ into the likelihood, and sample on $\btheta$,
that is we consider
\begin{align}
    \label{eq:injection-bayes-theorem}
    P\left(\btheta|\bx_{0,n}\right)
    =
    \frac{P\left(\bx_{0,n}|\btheta\right)P\left(\btheta\right)}{P\left(\bx_{0,n}\right)}
    .
\end{align}
The probability distribution $P\left(\btheta|\bx_{0,n}\right)$ gives us an understanding
of how well we could measure $\btheta_0$, given (near, if $\bn\neq0$) optimal data.
Moreover, it can give us an idea of how the different components of $\btheta$ may
be correlated with one another.

To determine $P\left(\btheta|\bx\right)$, we either need to directly
sample $\btheta$ from \eqref{eq:injection-bayes-theorem}, or use an approximation method.
%===============================================================================
\section{Fisher forecasting}

Injection analysis with a multivariate normal approximation for the likelihood is called
\textbf{Fisher forecasting}.
Let $\hat{\btheta}$ be the maximum likelihood estimator for $\btheta$.
Under appropriate regularity conditions, in the limit of a large number of observations,
$P\left(\bx|\hat{\btheta}\right)$ tends towards the normal
distribution (in parameter space), with mean $\hat{\btheta}$ 
and covariance matrix 
\begin{align}
    \label{eq:bvm-thm-cov}
    \Sigma^{(F)}_{ij}= \frac{1}{N}F^{-1}_{ij}\left(\hat{\btheta}\right)
    .
\end{align}
Here $N$ is the number of observations.
This is known as the \textbf{Bernstein–von Mises theorem}. 
We provide a proof of this result in Appendix \ref{chap:probability-theory}.

Estimating the maximum likelihood estimator $\hat{\btheta}$ is a challenging task of its own, 
that we do no explore further here.
Using the Bernstein-von Mises theorem, the posterior probability distribution near $\hat{\btheta}$, 
in the limit of a large number of observations, is approximately 
\begin{align}
    P\left(\btheta|\bx\right)
    = 
    \frac{\pi\left(\btheta\right)}{\mathcal{Z}\left(\bx,M\right)}
    \exp\left[
        -
        \frac{1}{2}
        \left(\theta-\hat{\theta}\right)^i
        F_{ij}
        \left(\theta-\hat{\theta}\right)^j
    \right]
    .
\end{align}
If we assume a Gaussian prior on the parameters $\btheta$, then the
posterior is a multivariate Gaussian with an inverse covariance matrix given by
\begin{align}
    \label{eq:Fisher-injection-inverse-covariance}
    \Sigma_{ij}^{-1} 
    = 
    N F_{ij} 
    + 
    \frac{1}{\sigma_i^2}\delta_{ij} 
    .
\end{align}
That is, the posterior probability distribution within this approximation is
\begin{align}
    P\left(\btheta|\bx\right)
    \approx
    \frac{1}{\sqrt{\left(2\pi\right)^k\det\Sigma}}
    \exp\left[
        -
        \frac{1}{2}
        \left(\theta-\hat{\theta}\right)^i
        \Sigma^{-1}_{ij}
        \left(\theta-\hat{\theta}\right)^j
    \right]
    ,
\end{align}
where $k$ is the dimensionality of $\btheta$, that is the number of parameters.

To perform a \textbf{Fisher forecast} for a given model $P\left(\bx|\btheta\right)$, 
we pick a set of parameters $\btheta_0$, and then compute the Fisher information matrix
\eqref{eq:Fisher-information}.
That is, we assume that $\btheta_0$ are the ``true'' model parameters, 
and also are the maximum likelihood estimators.
We then ``inject'' those parameters into the likelihood, which we approximate as a
multivariate Gaussian with inverse covariance matrix given by 
\eqref{eq:Fisher-injection-inverse-covariance}.
This analysis can be useful to determine the strength of correlation between different
the different components of $\theta^i$ (through the off-diagonal terms in $\Sigma_{ij}$).
The diagonal of the covariance matrix additionally gives us the 1-$\sigma$
error bars of the parameters.
If we could make $N$ measurements of the same data, each element in the
covariance matrix would decrease $1/N$, as follows from \eqref{eq:bvm-thm-cov}.
We see that the Fisher matrix can also give us a rough estimate of the number
of observations $N$ that are needed to make a $n-\sigma$ observation of
a parameter $\theta^i$.

Fisher forecasting is sometimes said to provide an optimal estimate of the
variance of the parameters in a given measurement.
This statement is justified by the \textbf{Cram\'{e}r-Rao bound}, which states
that the covariance matrix of an unbiased estimator for $\btheta$, $\bTheta$, 
(that is $\mathbb{E}\left[\bTheta\left(\bx\right)\right]=\btheta$)
is bounded from below by the inverse of the Fisher information matrix
\begin{align}
    \label{eq:cramer-rao-unbiased}
    \Sigma_{ij}\Big|_{\btheta=\bmu_{\btheta}} 
    \geq 
    F_{ij}^{-1}\left(\btheta\right)
    .
\end{align}
This bound should be interpreted with caution though, as \eqref{eq:cramer-rao-unbiased}
only holds for unbiased estimators to the parameters $\btheta$.
Consider a general estimator $\bTheta\left(\bx\right)$, and denote its expectation by
\begin{align}
    \mathbb{E}\left[\bTheta\left(\bx\right)\right]
    =
    \bpsi\left(\btheta\right)
    .
\end{align}
The Cram\'{e}r-Rao bound states that
\begin{align}
    \label{eq:cramer-rao-biased}
    \nabla_{\theta_m}\psi_i
    \nabla_{\theta_n}\psi_j
    F_{mn}^{-1}\left(\btheta\right) 
    .
\end{align}
If $\bTheta$ is an unbiased estimator ($\bpsi=\btheta$), then
\eqref{eq:cramer-rao-biased} reduces to \eqref{eq:cramer-rao-unbiased}.
We outline a proof of \eqref{eq:cramer-rao-biased} in Appendix \ref{chap:probability-theory}.
