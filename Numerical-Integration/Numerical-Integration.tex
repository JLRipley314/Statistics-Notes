We we discussed in Sec.~\ref{chap:bayesian-overview}, 
in parametric Bayesian statistics
our goal is to determine the posterior probability distribution of the
parameters of the model under consideration, given a set of measured data,
or to determine the total evidence for the model. 

A large portionof omputational, parameteric Bayesian statistics essentially consists of determining ways to compute high dimensional integrals.
To understand why we need to compute integrals in parameteric Bayesian statistics 
we look again at Bayes theorem
\begin{align}
    \label{eq:Bayes-theorem}
    P\left(\btheta|\bx\right)
    =
    \frac{
        \mathcal{L}\left(\btheta\right)\pi\left(\btheta\right)
    }{
        \mathcal{Z}\left(x\right)
    }
    ,
\end{align}
where $\pi$ is the prior and the the likelihood $\mathcal{L}$ and
evidence $\mathcal{Z}$ are
\begin{align}
    \label{eq:likelihood}
    \mathcal{L}\left(\btheta\right)
    =&
    \prod_{i=1}^NP\left(\bx_i|\btheta\right)
    ,\\
    \label{eq:evidence}
    \mathcal{Z}\left(\btheta\right)
    =&
    \int d^k\theta 
        \mathcal{L}\left(\btheta\right)\pi\left(\btheta\right)
    .
\end{align}
We have assumed the measurements of $\bx$ have been taken idependently
on one another in the equation for the likelihood.
We assume that the parameters $\btheta$ are continuous.
Clearly if we want to determine $P\left(\btheta|\bx\right)$ directly, we need to compute the evidence $\mathcal{Z}$, which requires integrating over the likelihood.
Beyond this though, many summary statistics of practical interest require computing an integral.
For example, we may be interested in the expectation of $\btheta$ and its covariance matrix 
\begin{subequations}
\begin{align}
    \mu_i
    =
    \mathbb{E}\left[\theta_i\right]
    &\equiv
    \int d^k\theta P\left(\btheta|\bx\right)\theta_i 
    ,\\
    C_{ij}
    \equiv
    \mathbb{E}\left[\left(\theta_i-\mu_i\right)\left(\theta_j-\mu_j\right)\right]
    &\equiv
    \int d^k\theta P\left(\btheta|\bx\right)
        \left(\theta_i-\mu_i\right)\left(\theta_j-\mu_j\right) 
    ,
\end{align}
\end{subequations}
We cover methods to compute high dimensional integrals, as in many applications $k\gg1$ (or at least, $k\gtrsim10$).
In this regime, it is usually computationally infeasible to compute \eqref{eq:evidence} using traditional deterministic methods such as the trapezoid rule or Gaussian quadrature.
For example if $k=10$, and if we have $10$ quadrature points in each parameter direction, we will need to make $N\gtrsim 10^{10}$ evaluations for a trapsoid rule approximation ot the evidence.
The likelihood function is often a highly complex function with sharp peaks,
so many more than $10$ grid points would be needed to resolve in each direction
in order to properly resolve the posterior.

As far as I am aware, the most efficient way to compute high dimensional
integrals is through stochastic/Monte Carlo methods.
The fact that Monte Carlo methods are the best methods to compute many high
dimensional integrals is somewhat surprising, as they have very slow
rates of convergence.
In general, the error of Monte Carlo integrals goes as 
$N^{-1/2}$, where $N$ is the number of points used in the approximation.
In one dimension, approximations as simple as the trapezoid rule converge
to the correct answer as $1/N^2$ (e.g. \cite{PresTeukVettFlan92}).
This being said, the accuracy of methods such as the trapezoid rule rapidly
deteriorate at higher dimension, while for Monte Carlo methods, the accuracy
decreases as $N^{-1/2}$, regardless of the dimensionality of the problem,
althouh the proportionality constant to this decrease strongly depends on the choice of algorithm one uses, and the dimensionality of the problem. 
We only consider stochastic/Monte Carlo integration methods in this chapter.

In effect, Bayesian parametric statistics reduces statistics to probability
theory, and many problems in probability theory can be reduced to problems in
the integration of complciated functions in high dimensional spaces.
There are three main approaches to integration, 
\textbf{Riemann integration}, \textbf{Riemannian-Stieltjes integration},
and \textbf{Lebesgue integration}.

\textbf{Monte Carlo integration} can be thought of as providing an
approximation to the Riemannian integral. 
We review Monte Carlo integration in Sec.~\ref{eq:Monte-Carlo-integration}.
The most commonly used variant of Monte Carlo integration is
\textbf{Markov chain Monte Carlo} (MCMC) integration, 
which can be thought of as approximating the Riemann-Stieltjes integral. 
We eview MCMC integration in Sec.~\ref{sec:mcmc}.
The Monte Carlo approximation of certain kinds of Lebesgue integrals goes under the name \textbf{Nested Sampling} (NS), which we review in Sec.~\ref{sec:nested-sampling}.
There are many excellent, long discussions of all these methods on the
internet and elsewhere
(e.g. \cite{brooks2011handbook,skilling-nested-sampling,Hogg:2017akh}), 
so we only outline the main ideas.

Before continuing, we mention two applications where you
do not need to compute an integral
(and hence do not need to use the methods discussed here).
If we only need to compute the ratio of the posterior for two parameters
values $\btheta_1$ and $\btheta_2$,
we only need to determine
$P\left(\btheta_1|\bx\right)/P\left(\btheta_2|\bx\right)
=\mathcal{L}\left(\btheta_1\right)/\mathcal{L}\left(\btheta_2\right)$,
which does not involve any integrals.
We also do not need to compute any integrals if we only want the maximum
of the posterior or the likelihood (the maximum likelihood estimator).
We review some maxization methods in  Chptr.~\ref{chap:numerical-optmization}.


%==============================================================================
\section{Monte Carlo integration\label{eq:Monte-Carlo-integration}}

Consider a function $f\left(\btheta\right)$, and an integral over the domain $\Omega$
\begin{align}
    \label{eq:integral-f-monte-carlo}
    I
    =
    \int_{\Omega} d^k\theta f\left(\btheta\right)
    .
\end{align}
We can view \eqref{eq:integral-f-monte-carlo} as the expectation of $f$ 
over $\Omega$, with respect to the uniform distribution $U\left(\Omega\right)$.
In Monte-Carlo integration, we sample points uniformly on $\Omega$, and then
approximate $I$ via
\begin{align}
    \label{eq:monte-carlo-integral-f}
    I_N
    = 
    \frac{1}{N}\sum_{i=1}^N f\left(\bx_i\right)
    .
\end{align}
Here $N$ is the number of times we have sampled from $U\left(\Omega\right)$,
and $\bx_i$ are the sample points.
Monte Carlo integration works if we can efficiently evaluate $f$. 
From the law of large numbers,
\begin{align}
    \lim_{N\to\infty}I_N = I
    .
\end{align}
The standard error of the mean goes as $N^{-1/2}$, which gives us our estimate
for the error of this approximation; that is we can write (e.g. \cite{PresTeukVettFlan92})
\begin{align}
    \int_{\Omega} d^k\theta f\left(\btheta\right)
    \approx
    V\left(\Omega\right)
    \left(
        \mathbb{E}\left[f\right]
        \pm
        \sqrt{\frac{\mathbb{V}\left[f\right]}{N}}
    \right)
    ,
\end{align}
where $V\left(\Omega\right)$ is the volume of $\Omega$.
We see that the convergence of Monte Carlo integration scales as $1/\sqrt{N}$,
regardless of the dimensionality of the integral.
This is the key property of stochastic integration methods, and what makes them
widespread use in computing high dimensional integrals.
In one dimension, almost any other quadrature method outperforms Monte Carlo integration 
(for example, the error to the trapezoid rule scales as $1/N^2$),
but for higher dimensional integrals the convergence of most methods rapidly deteriorates.

We can think of Monte Carlo integration as an example of a stochastic approximation
to the Riemann integral of $f$.
Recall that the Riemann integral is the limit of the sum over $f(\btheta_i)$
multiplied by the volume of a small (possibly multidimensional) 
rectangle centered on $f\left(\btheta_i\right)$,
which we call $V\left(\btheta_i\right)$.
\begin{align}
    \int d^k\theta f\left(\btheta\right)
    =
    \lim_{N\to\infty}\sum_{i=1}^N V\left(\btheta_i\right)f\left(\btheta_i\right)
    .
\end{align}
In effect, in Monte Carlo integration we approximate $V\left(\btheta_i\right)$
with $n/N$, where $n$ is the number of draws we made in that volume.

Monte Carlo integration works best if most of the integral $I$ is not concentrated in a few small volume regions; that is if $f\left(\btheta\right)$ is not too ``peaked''.
Most likelihoods are strongly peaked though--for example from the
Bernstein–von Mises theorem (see Appendix \ref{chap:probability-theory})
we expect the likelihood to approximately behave as a multivariate normal function 
around the maximum likelihood estimator as the amount of data we collect
goes to infinity. 
Moreover the covariance matrix elements scale as $1/N_d$, where $N_d$ is the
number of data points, so the distribution becomes increasingly localized
near the maximum likelihood estimator, and more generally near other local
maxima of the likelihood.
That is, selecting $\btheta_i$ from the uniform distribution could mean that
we are mostly sampling from places where $f\left(\btheta_i\right)$ is much smaller
than near the peaks. 
In that case, we would be missing most what contributes to the integral
\eqref{eq:integral-f-monte-carlo}, which slows down the rate of convergence
(the prefactor in front of the asymptotic scaling of $1/\sqrt{N}$).

This motivates the introduction of integrations methods that preferentially sample
from regions near the local maxima of the integrand in 
\eqref{eq:integral-f-monte-carlo}.
We next discuss two such adaptive methods:
Markov chain Monte Carlo (MCMC) methods, which can
also be thought of as an adaptive approximation to the Riemann integral, and
nested sampling methods, which can be though of as an adaptive approximation 
to the Lebesgue integral\footnote{For smooth functions--which is what the posterior distribution
function $P\left(\btheta|\bx\right)$ is, provided the prior and 
our model $P\left(\bx|\btheta\right)$ are smooth--there 
is no substantive, practical difference between the Riemann and Lebesgue integral.
Nevertheless we will see that there are different strengths and weaknesses
to MCMC and nested sampling, 
unrelated to the kinds of integrals they are approximating.} 

%==============================================================================
\section{Markov chain Monte Carlo (MCMC)\label{sec:mcmc}}

The idea behind MCMC integration 
is to generate the random samples for the Monte Carlo
integration of $\Omega$ dynamically, through a Markov Chain. 
To do this, we rewrite the integral \eqref{eq:integral-f-monte-carlo} as follows
\begin{align}
    \label{eq:MCMC-integral}
    I
    =
    \int d^k\theta p\left(\btheta\right)g\left(\btheta\right) 
    ,
\end{align}
where
\begin{align}
    \int d^k\theta p\left(\btheta\right)
    =
    1
    .
\end{align}
That is, we interpret $p\left(\btheta\right)$ as a probability distribution.
We can view \eqref{eq:MCMC-integral} as a Riemann-Stieltjes integral,
\begin{align}
    \label{eq:riemann-steiltjes-integral}
    I
    =
    \int dF\left(\btheta\right) g\left(\btheta\right)
    ,
\end{align}
with the measure $dF\left(\btheta\right) \equiv d^k\theta p\left(\btheta\right)$.
We defined
\begin{align}
    \label{eq:F-RS-int}
    F\left(\lambda\right)
    \equiv
    \int_0^{g\left(\btheta\right)<\lambda} d^k\theta p\left(\btheta\right)
    .
\end{align}

Properly speaking, MCMC is a method for drawing samples from $p\left(\btheta\right)$, for use in calculating integrals of the form \eqref{eq:MCMC-integral}.
It turns out that a histogram of our sampling of $p\left(\btheta\right)$ will
begin to resemble $p\left(\btheta\right)$ as the number of draws goes to infinity.
For this reason, MCMC methods are often seen as ways to determine the ``shape''
or functional properties of $p\left(\btheta\right)$.
Here we take the perspective of numerical integration theory, so we think of 
$p\left(\btheta\right) = f\left(\btheta\right)/g\left(\btheta\right)$ 
as a weighting factor for our integration of \eqref{eq:integral-f-monte-carlo}. 

First we define a few concepts about Markov Chains.
We will borrow some of the concepts from the discussion on stochastic processes
in Chptr.~\ref{chap:time-series-analysis}.
A \textbf{Markov chain} is a sequence of random vectors
(a stochastic process) $\left\{\btheta_1,\btheta_2,...\right\}$, 
where the probability distribution of $\btheta_{n+1}$ is dependent solely on $\btheta_n$.
That is 
$P\left(\btheta_{n+1}|\left\{\btheta_i\right\}\right) 
 =
 P\left(\btheta_{n+1}|\btheta_n\right)$. 
We call $P\left(\btheta_{n+1}|\btheta_n\right)$ the \textbf{transition probability}. 
The probability distribution for $\btheta_1$, $P\left(\btheta_1\right)$, is called
the \textbf{initial distribution}.
We will only consider Markov chains with \textbf{stationary transition probabilities},
where the transition probabilities do not depend on $n$.

Operationally, MCMC integration of \eqref{eq:integral-f-monte-carlo}
goes as follows.
\begin{enumerate}
    \item We pick an initial point $\btheta_1$, and then generate new samples
        $\btheta_n$ based on a suitably chosen transition probability.
    \item For the first few iterations of the Markov Chain, the points $\btheta_n$
        will be highly correlated with our initial start point, but if one runs the
        Markov Chain for enough iterations, the points $\{\btheta_n\}$ will eventually
        converge to the target distribution $p\left(\btheta\right)$.
    \item Integration then proceeds as in Monte Carlo integration
        \begin{align}
            \label{eq:mcmc-integral-sum}
            I_N
            =
            \frac{1}{N}\sum_{i=1}^N g\left(\btheta_i\right)
            ,
        \end{align}
        with similar convergence properties to Monte Carlo integration 
        (the error will asymptotically go does as $1/\sqrt{N}$).
        The hope though is that the prefactor to the leading asymptotic decay will be
        much smaller than it would be for regular Monte Carlo integration.
\end{enumerate}
We see that we can view \eqref{eq:mcmc-integral-sum} as an approximation to the
Riemann–Stieltjes integral \eqref{eq:riemann-steiltjes-integral}.

There are whole volumes on MCMC (e.g. \cite{brooks2011handbook}); 
for a nice shorter review see \cite{Hogg:2017akh}.
Here we only outline what a ``suitable'' Markov Chain transition probability
must satisfy, the Metropolis-Hastings algorithm, and some limitations
of most MCMC methods.

An MCMC chain must eventually limit to a stationary distribution that is
equal to $p\left(\btheta\right)$.
A sufficient (but not necessary) conditions for a Markov chain to have a 
stationary distribution $Q\left(\btheta\right)$ is that the
transition probabilities must satisfy the \textbf{detailed balance} condition 
\begin{align}
    P\left(\btheta|\bpsi\right)Q\left(\bpsi\right)
    =
    P\left(\bpsi|\btheta\right)Q\left(\btheta\right)
    ,
\end{align}
for any $\btheta,\bpsi$. 
To see why detailed balance implies stationarity, we compute the probability
of a transition to new step $\btheta_n$. 
The probability distribution for a new step $\btheta$ is equal to the
integral (or sum, if there a discrete number of points)
over all possible earlier points $\psi$.
We assume those are distributed according to the probability distribution
$Q\left(\bpsi\right)$. We then show that the distribution for $\btheta$,
$P\left(\btheta\right)$, is equal to $Q\left(\btheta\right)$, which
implies that the chain is stationary. We have
\begin{align}
    P\left(\btheta\right)
    &=
    \int d^k\psi P\left(\btheta|\bpsi\right) Q\left(\bpsi\right)
    \nonumber\\
    &=
    \int d^k\psi P\left(\bpsi|\btheta\right) Q\left(\btheta\right)
    \nonumber\\
    &=
    \frac{Q\left(\btheta\right)}{P\left(\btheta\right)}
    \int d^k\psi P\left(\bpsi,\btheta\right)
    \nonumber\\
    &=
    Q\left(\btheta\right)
    .
\end{align}
This proves existence of a stationary chain, but it does not prove uniqueness.
Proving uniqueness of the stationary distribution is beyond the scope of these notes.
Most practitioners simply ignore the question of uniqueness.

Finally, we discuss an example of a Markov Chain that satisfies 
the detailed balance
condition for the target function $p\left(\btheta\right)$
(lower case $p$; see \eqref{eq:MCMC-integral}):
the \textbf{Metropolis-Hastings algorithm}.
Consider a point $\btheta$.
We draw $\bpsi$ from the \textbf{proposal probability} $Q\left(\bpsi|\btheta\right)$
(we are free to specify $Q$).
We then draw a random variable $x$ from the uniform distribution $U\left(0,1\right)$. We next compute the \textbf{acceptance probability} 
\begin{align}
    \label{eq:metropolis-p}
    r 
    =
    \min\left(
        1,
        \frac{p\left(\btheta\right)}{p\left(\bpsi\right)}
        \frac{Q\left(\btheta|\bpsi\right)}{Q\left(\bpsi|\btheta\right)}
    \right)
    .
\end{align}
If $x>r$, we jump to the point $\bpsi$, otherwise, we stay at the point $\btheta$.
To prove that transition probability in the 
Metropolis-Hastings algorithm satisfies the detailed balance condition, 
we rewrite the transition probability amplitude as being equal to
the proposal probability times the acceptance probability
\begin{align}
    P\left(\bpsi|\btheta\right)
    =
    r \times Q\left(\bpsi|\btheta\right)
    .
\end{align}
We then have
\begin{align}
    P\left(\btheta|\bpsi\right)p\left(\bpsi\right)
    &=
    \min\left(
        p\left(\bpsi\right)Q\left(\bpsi|\btheta\right)
        ,
        p\left(\btheta\right)Q\left(\btheta|\bpsi\right)
    \right)
    \nonumber\\
    &=
    \min\left(
        p\left(\btheta\right)Q\left(\btheta|\bpsi\right)
        ,
        p\left(\bpsi\right)Q\left(\bpsi|\btheta\right)
    \right)
    \nonumber\\
    &=
    P\left(\bpsi|\btheta\right)p\left(\btheta\right)
    .
\end{align}

%==============================================================================
\section{Nested sampling\label{sec:nested-sampling}}

As with MCMC integration, we consider integrals of the form
\begin{align}
    \label{eq:integral-nested-ptheta}
    I
    =
    \int d^k\theta p\left(\btheta\right) L\left(\btheta\right)
    .
\end{align}
We can only integrate positive definite functions with the nested sampling algorithm, 
which is why we use the slightly different notation of $L$ instead of $g$ here: 
we restrict to functions such that $L\geq0$.
This notation is motivated from the following:
the main application of the nested sampling integration is to compute the evidence 
$\mathcal{Z}$,
which is an integral of the prior probability distribution times the likelihood
\begin{align}
    \mathcal{Z}
    =
    \int d^k\theta \pi\left(\btheta\right) \mathcal{L}\left(\btheta\right)
    .
\end{align}

Before we describe the algorithm, we first need to rewrite \eqref{eq:integral-nested-ptheta}
as an integral over the level sets of $L\left(\btheta\right)$.
To do this, we write \eqref{eq:integral-nested-ptheta} as Riemann-Stieltjes integral
\eqref{eq:riemann-steiltjes-integral}, and then integrate by parts.
We define the function
\begin{align}
    \label{eq:F-NS-int}
    X\left(\lambda\right)
    &\equiv
    \int_{L\left(\btheta\right)>\lambda} d^k\theta p\left(\btheta\right)
    .
\end{align}
As $\lambda$ increases, $X$ decreases from $1$ to $0$.
With this, we have
\begin{align}
    \label{eq:integration-by-parts}
    I
    =
    \int d^k \theta p\left(\btheta\right) L\left(\btheta\right)
    =&
    -
    \int dX L
    \nonumber\\
    =&
    -
    X\left(L\right) L\big|_{L=0}^{L=L_{max}}
    +
    \int_0^{L_{max}} dL X\left(L\right)
    \nonumber\\
    =&
    \int_0^{L_{max}} dL X\left(L\right)
    .
\end{align}
We assumed that $L_{min}=0$ (which holds for the likelihood function), and used that $X=0$ at $L=L_{max}$. 
We assume that we can invert $X\left(L\right)$ to the function $L\left(X\right)$.
We can then rewrite \eqref{eq:integration-by-parts} by integrating by parts, to obtain
\begin{align}
    \label{eq:integral-nested-X}
    I
    =
    -
    \int_1^0 dX L\left(X\right)
    =
    \int_0^1 dX L\left(X\right)
    .
\end{align}
Unlike \eqref{eq:integral-nested-ptheta}, \eqref{eq:integral-nested-X}
is a one-dimensional integral. 
In \eqref{eq:integral-nested-X} we should think of $L$ as the parameter
in $X\left(L\right)$, not as $L\left(\btheta\right)$.
Nested sampling provides a noisy approximation to \eqref{eq:integral-nested-X},
through a partitioning of the $X$ interval, and hence provides a noisy approximation
to the Lebesgue integral of \eqref{eq:integral-nested-ptheta}.

To understand why \eqref{eq:integral-nested-X} 
is the Lebesgue integral of \eqref{eq:integral-nested-ptheta},
recall that the Lebesgue integral is the limit of the sum over 
$g_i\equiv g\left(\btheta_i\right)$
multiplied by the Lebesgue measure of the set $E_i$ of points $\btheta_j$ for which
$g\left(\btheta_j\right)\approx g\left(\btheta_i\right)$
\begin{align}
    \label{eq:lebesgue-sum}
    I_N
    =
    \sum_{i=1}^N g_i \mu\left(E_i\right)
    .
\end{align}
We can think of $g_i \mu\left(E_i\right)$ as the discretization of $dX \lambda\left(X\right)$.

The nested sampling algorithm goes as follow
\begin{enumerate}
    \item We draw $n$ points $\btheta_i$ from $p\left(\btheta\right)$, treating
        it as a probability distribution. Set $X_0=1$.
    \item Repeat for $N$ times, so you have the sequence $X_1,...,X_n$ and
        $L_{min,1},...,L_{min,N}$. For the $j^{th}$ iteration
    \begin{enumerate}
        \item Record the lowest value of $=L_{min,j}=L\left(\btheta_j\right)$. 
            Set $X_j = e^{-j/n}$, or alternatively set $X_j = t_j X_{j-1}$, where
            $t_j$ is drawn from the beta distribution $\mathrm{Beta}\left(1,n\right)$.
        \item Remove the value of $\btheta_j$ that minimizes $L\left(\btheta\right)$,
            and then sample again from $p\left(\btheta\right)$, until you get a point
            $\btheta_k$ such that $L\left(\btheta_k\right)>L_{min,j}$.
    \end{enumerate}
    \item The integral can then be obtained by summing the $L$ values via some quadrature rule 
        \begin{align}
            I_N
            =
            \sum_{i=1}^{N-1} w_i f\left(L_{min,i}\right)
            .
        \end{align}
        For example, for the trapezoid rule we would set $w_i = X_{i}-X_{i-1}$ and $f\left(L_{min,i}\right)=\left(L_{min,i} + L_{min,i-1}\right)/2$.  
\end{enumerate}
In the context of Bayes rule, where $I_N$ is our estimate of the evidence $\mathcal{Z}$, we can obtain an estimate for the posterior probability distribution via the rule
\begin{align}
    p\left(\btheta_i\right)
    \approx
    \frac{w_i f\left(L_{min,i}\right)}{I_N}
    .
\end{align}

One of the tricky things to understand about the nested sampling method is the value of the measure of the likelihood $L_{min,i}$, $X_i$.
Consider a sample from $p\left(\btheta\right)$: $\left\{\theta_j\right\}$, subject to
$L\left(\btheta_j\right)>L_{min,j-1}$. 
So long as we draw $\btheta_j$ from $p\left(\btheta\right)$
subject to to the constraint $L\left(\btheta\right)>L_j$,
the values of the volumes $X\left(\btheta_j\right)$ are drawn from $U\left(0,X_{j-1}\right)$.
This follows from the \textbf{probability integral transform}, which we review
in Chptr.~\ref{chap:probability-theory}.
Then $X_j = t_j X_{j-1}$, where $t_j$ is the largest of $n$
uniformly distributed numbers in the interval $(0,1)$.
The number $t_j$ is called the \textbf{shrinkage factor}.
Notice that we have 
\begin{align}
    \label{eq:shrinkage-product}
    X_j = \prod_{i=1}^{j} t_i 
    .
\end{align}
The cumulative probability distribution function for the maximum of $n$ randomly
distributed numbers in that interval is
\begin{align}
    C.D.F.\left(t_{max}\right) 
    &=
    P\left(\max\left\{t_1,...,t_n\right\}<t_{max}\right)
    \nonumber\\
    &=
    \left(P\left(t<t_{max}\right)\right)^n
    \nonumber\\
    &=
    t_{max}^{n}
    .
\end{align}
The probability density function for the maximum is then the beta distribution 
$\mathrm{Beta}\left(1,n\right)$, that is
\begin{align}
    P\left(t_{max}\right) 
    =
    n t_{max}^{n-1}
    .
\end{align}
To estimate $X_j$ then, we could take a draw from the Beta distribution, $t$,
and multiply that by $X_{j-1}$.
To get a (presumably) less noisy answer, we could set $X_j$ to be its averaged
expected value.
We take the expectation of the log of $X_j$, to simplify the calculation of the expectation
\begin{align}
    \mathbb{E}\left[\log X_j\right]
    =
    \sum_{i=1}^j \mathbb{E}\left[\log t_i\right]
    .
\end{align}
As the $t_i$ are independent, we can estimate the error of this approximation by 
computing the variance
\begin{align}
    \mathbb{V}\left[\log X_j\right]
    =
    \sum_{i=1}^j \mathbb{V}\left[\log t_i\right]
    .
\end{align}
The expectation value of the logarithm of $t_{max}$ is
\begin{align}
    \mathbb{E}\left[\log t_{max}\right]
    =&
    \int_0^1 dt \;n t^{n-1}\;\log t 
    \nonumber\\
    =&
    -
    \frac{1}{n}
    .
\end{align}
The variance of the log of $t_{max}$
\begin{align}
    \mathbb{V}\left[\log t_{max}\right]
    =&
    \mathbb{E}\left[\left(\log t_{max}\right)^2\right]
    -
    \left(\mathbb{E}\left[\log t_{max}\right]\right)^2
    \nonumber\\
    =&
    \int_0^1 dt \; n t^{n-1} \; \left(\log t\right)^2
    -
    \frac{1}{n^2}
    \nonumber\\
    =&
    \frac{1}{n^2}
    .
\end{align}
Combining everything, we see that the shrinkage factor is approximately
\begin{align}
    \log X_j
    \approx&
    j \; \mathbb{E}\left[ \log t_{max}\right]
    +
    \sqrt{j \; \mathbb{V}\left[\log t_{max}\right]}
    \nonumber\\
    =&
    -
    \frac{j}{n}\left(1 \pm \frac{1}{\sqrt{j}}\right) 
    .
\end{align}
This gives us
\begin{align}
    X_j
    \approx 
    e^{-j/n}
    .
\end{align}
Our approximation to the integral gets better as we add more points $n$,
and as we take more steps $N$. 
We incur the biggest relative errors in the integration for the first few small
steps $j$, but so long as $L$ is highly peaked, and we take very small steps,
those terms contribute very little to the total integral.
