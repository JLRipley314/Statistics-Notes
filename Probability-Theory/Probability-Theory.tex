%--------------------------------------------------------------------
\section{Note on notation}

We denote the probability of measuring $\btheta$ by $P\left(\btheta\right)$,
the joint probability by $P\left(\btheta,\bx\right)$, 
and the conditional probability by $P\left(\btheta|\bx\right)$.
In probability theory it is common to denote a probability distribution
function (PDF) as $f_{\bX}\left(\bx\right)$ for a random variable $\bX$.
To get a probability from a continuous PDF, you need to perform an integral.
In other words, technically $f_{\bX}\left(\bX\right)$ is not a probability,
but $\int_{V} d^nx f_{\bX}\left(\bX\right)$ over some volume $V$ is a probability.
In situations where we may be refering to either a PDF or a probability,
we will simply use $P$, which hopefully won't be too confusing. 

When a random variables $\bX$ has a PDF $f_{\bX}\left(\bx|\btheta\right)$,
we say $\bX \sim f\left(\btheta\right)$, where $f$ denotes that probability
and $\btheta$ are the hyperparameters of the model.

%--------------------------------------------------------------------
\section{Conditional probability and Bayes theorem}

The conditional probability is
\begin{align}
    P\left(\btheta|\bx\right)
    =
    \frac{P\left(\btheta,\bx\right)}{P\left(\bx\right)}
    .
\end{align}
Using this, we have Bayes theorem
\begin{align}
    \label{eq:Bayes-theorem-1}
    P\left(\btheta|\bx\right)P\left(\bx\right)
    =
    P\left(\bx|\btheta\right)P\left(\btheta\right)
    .
\end{align}
Bayes theorem is more often written as
\begin{align}
    \label{eq:Bayes-theorem-2}
    P\left(\btheta|\bx\right)
    =
    \frac{
        P\left(\bx|\btheta\right)P\left(\btheta\right)
    }{
        P\left(\bx\right)
    }
    .
\end{align}
As we discuss more in Chptr.~\ref{chap:bayesian-overview}, 
we can view $P\left(\btheta|\bx\right)$ as the distribution of
model parameters given a model $P\left(\bx|\btheta\right)$,
and some prior knowledge of the model parameters $P\left(\btheta\right)$.
%--------------------------------------------------------------------
\section{Cumulative distribution function}

The \textbf{cumulative distribution function} (CDF) $F$ for a probability distribution function $f_{\bTheta}$ is
\begin{align}
    F_{\bTheta}\left(\xi\right)
    &=
    P\left(\btheta < \bF^{-1}_{\bTheta}\left(\xi\right)\right)
    \nonumber\\
    &=
    \int_{P\left(\btheta\right)<\xi} d^k\theta f_{\bTheta}\left(\btheta\right)
    ,
\end{align}
where
\begin{align}
    \bF^{-1}_{\bTheta}\left(\xi\right)
    \equiv
    \inf\left\{\btheta \; : \; F_{\bTheta}\left(\btheta\right) \geq \xi\right\}
    .
\end{align}
Note that $F_{\bTheta}\left(\bF^{-1}_{\bTheta}\left(\btheta\right)\right) = \btheta$.
Sometimes in the literature you'll see $F^{-1}_{\bTheta}\left(\xi\right)$ -- which we can efectively think of as the inverse of the CDF -- is called the \textbf{percent point function}.

The \textbf{probability integral transform} states that the random variable $\Xi$ defined to be
\begin{align}
    \Xi = F_{\bTheta}\left(\xi\right)
    ,
\end{align}
has the standard uniform distribution, that is $\Xi\sim U\left(0,1\right)$.
To prove this, we look at the CDF of $\xi$
\begin{align}
    F_{\Xi}\left(\xi\right)
    &=
    P\left(\Xi < \xi\right)
    \nonumber\\
    &=
    P\left(F_{\bTheta}\left(\btheta\right) < \xi\right)
    \nonumber\\
    &=
    P\left(\btheta < \bF^{-1}_{\bTheta}\left(\xi\right)\right)
    \nonumber\\
    &=
    F_{\bTheta}\left(\bF^{-1}_{\bTheta}\left(\xi\right)\right)
    \nonumber\\
    &=
    \xi
    .
\end{align}
We see that the CDF of $F_{\Xi}\left(\xi\right)$ is the same
as the CDF for the uniform distribution $U\left(0,1\right)$.
We conclude that $\Xi\sim U\left(0,1\right)$.
This fact is used in the nested sampling integration algorithm,
which we discussed in Chpt.~\ref{chap:numerical-integration}.

%--------------------------------------------------------------------
\section{Functions of random variables}

We consider two random (scalar) variables $X$ and $Y$, with the join PDF $f_{X,Y}\left(x,y\right)$.
Our goal is to find the PDF of the function $Z\left(X,Y\right)$ (generalizing to functions of a larger number of random variables is straighforward).
To find this, we first determine the set
\begin{align}
    A_{z}
    =
    \left\{\left(x,y\right) \; : \; Z\left(x,y\right) \leq z\right\}
    .
\end{align}
Given this, we can then compute the CDF
\begin{align}
    F_Z\left(z\right)
    =
    \int_{A_z} dx dy f_{X,Y}\left(x,y\right)
    .
\end{align}
From the CDF we can then compute the PDF by taking the derivative of the CDF
\begin{align}
    f_Z\left(z\right) = \frac{d}{dz}F_Z\left(z\right)
    .
\end{align}
%--------------------------------------------------------------------
\section{Change of variables for injective mappings}

If we restrict ourselves to injective mappings between random variables, we can derive a simple closed-form expression for the PDF of a variable $\bPsi\left(\bTheta\right)$.
Notice that we do not need to restrict ourselves to scalar random variables here.
Consider a PDF $f_{\bTheta}\left(\btheta\right)$.
What is the probability distribution to $P\left(\bpsi\left(\btheta\right)\right)$,
where $\bpsi$ is some function of $\btheta$?
The following remains unchanged under a change of variables
\begin{align}
    \int_V d^k\theta f_{\bTheta}\left(\btheta\right)
    &=
    \int_{V} d^k\psi f_{\bPsi}\left(\bpsi\right)
    \nonumber\\
    &=
    \int_{V} d^k\theta \left|\det\left(J_{ij}\right)\right| 
        f_{\bPsi}\left(\bpsi\right)
    .
\end{align}
We viewed $V$ as a geometric volume (that is, it is independent of the coordinate choice we use).
Here 
\begin{align}
    \label{eq:Jacobian-matrix}
    J_{ij}
    \equiv
    \frac{\partial\psi^i}{\partial\theta_j}
    ,
\end{align}
is the Jacobian matrix.
Equating terms within the integral, we find that
\begin{align}
    \label{eq:change-of-variable-pdf}
    f_{\bPsi}\left(\bpsi\right)
    =
    \frac{1}{\left|\det\left(J_{ij}\right)\right|} f_{\bTheta}\left(\btheta\right)
    .
\end{align}
As an example application of this formula, we consider the posterior probability distribution for $\bpsi\left(\btheta\right)$. 
If $\bpsi\left(\btheta\right)$ is not injective, we need to replace the RHS of \eqref{eq:change-of-variable-pdf} with a sum over the different values of $\btheta$ that map to the same $\bpsi$.
In other words, we have
\begin{align}
    \label{eq:change-of-variable-pdf}
    f_{\bPsi}\left(\bpsi\right)
    =
    \sum_{\btheta \; : \; \bpsi\left(\btheta\right)=\bpsi}
    \frac{1}{\left|\det\left(J_{ij}\right)\right|} f_{\bTheta}\left(\btheta\right)
    .
\end{align}
Note that in general this ``sum'' may in fact be an integral.
From Bayes theorem \eqref{eq:Bayes-theorem-2}, we then have
\begin{align}
    f_{\bPsi}\left(\bpsi|\bx\right)
    &=
    \frac{
        f_{\bX}\left(\bx|\bpsi\left(\btheta\right)\right)
        f_{\bPsi}\left(\bpsi\right)
    }{
        f_{\bX}\left(\bx\right)
    }
    \nonumber\\
    &=
    \frac{1}{\left|\det\left(J_{ij}\right)\right|}
    \frac{
        f_{\bX}\left(\bx|\bpsi\left(\btheta\right)\right)
        f_{\bTheta}\left(\btheta\right)
    }{
        f_{\bX}\left(\bx\right)
    }
    .
\end{align}
That is, to find the probability distribution for some function of the distribution parameters, we only need to find the probability distribution for the prior under that change in coordinates. 


%===============================================================================
\section{Conditional probability}

The conditional probability is defined by
\begin{align}
    P\left(\btheta,\bpsi\right)
    \equiv
    P\left(\btheta|\bpsi\right)
    P\left(\bpsi\right)
    .
\end{align}
For PDFs, a useful result is that
\begin{align}
    f_{\bTheta}\left(\btheta\right)
    =
    \int d^n\psi f_{\bTheta|\bPsi=\bpsi}\left(\bTheta\right)f_{\bPsi}\left(\bpsi\right)
    .
\end{align}

%===============================================================================
\section{Expectation and covariance}

We review a few basic definitions from probability theory, as they come
up later in the notes.
We define the \textbf{expectation} of a random variable $\bTheta\left(\bx\right)$ to be
\begin{align}
    \label{eq:expectation-def}
    \mathbb{E}\left[\bTheta\right]
    \equiv
    \int d^kx f_{\bX}\left(\bx\right) \bTheta\left(\bx\right)
    ,
\end{align}
where $f_{\bX}$ is the PDF of $\bx$.
Sometimes we denote the expectation with $\mu_{\bTheta}$.
The \textbf{variance} is
\begin{align}
    \label{eq:variance-def}
    \mathbb{V}_{ij}\left[\bTheta\right]
    \equiv
    \mathbb{E}\left[
        \left(\Theta_i-\mu_{\Theta_i}\right)
        \left(\Theta_j-\mu_{\Theta_j}\right)
    \right]
    .
\end{align}

The \textbf{covariance} for two random variables $\bTheta,\bPsi$ is
\begin{align}
    \label{eq:covariance-def}
    \mathbb{C}_{ij}\left[\bTheta,\bPsi\right]
    \equiv
    \mathbb{E}\left[
        \left(\Theta_i - \mu_{\Theta_i}\right)
        \left(\Psi_j   - \mu_{\Psi_j}  \right)
    \right]
    .
\end{align}
Note that we can think of the expectation of two scalar random variables $\Theta,\Psi$ as an inner product
\begin{align}
    \label{eq:expectation-inner-product-def}
    \mathbb{E}\left[\Theta\Psi\right]
    =
    \left<\Theta,\Psi\right>
    .
\end{align}
It is easy to see from \eqref{eq:expectation-def} that \eqref{eq:expectation-inner-product-def} satisfies
the properties of an inner product: 
$\left<\Theta,\Theta\right>\geq0$, $\left<\Theta,\Psi\right>=\left<\Psi,\Theta\right>$, and linearity.

Consider a random variable $\bTheta\left(\bpsi\right)$. 
Say we want this variable to represent another random variable, say the parameters of the posterior $\btheta$.
We then call $\bTheta$ an \textbf{estimator} for $\btheta$.
The \textbf{bias} of $\bTheta$ then is
\begin{align}
    \label{eq:bias-def}
    \bb\left(\bTheta\right)
    \equiv
    \mathbb{E}\left[\bTheta\right]
    -
    \btheta
    .
\end{align}
If $\bb=0$, then $\bTheta$ is an \textbf{unbiased estimator} for $\btheta$.

The \textbf{law of total expectation} states that
\begin{align}
    \label{eq:law-total-expectation}
    \mathbb{E}\left[\bTheta\right]
    &=
    \mathbb{E}\left[\mathbb{E}\left[\bTheta|\bPsi\right]\right]
    .
\end{align}
This follows from the definition of the expecation
\begin{align}
    \mathbb{E}\left[\mathbb{E}\left[\bTheta\right]\right]
    &=
    \int d^n\psi\mathbb{E}\left[\bTheta|\bPsi=\bpsi\right] f_{\bPsi}\left(\bpsi\right)
    \nonumber\\
    &=
    \int d^kx\int d^n\psi \bTheta\left(\bx\right) f_{\bX|\bPsi=\bpsi}\left(\bx\right) f_{\bPsi}\left(\bpsi\right)
    \nonumber\\
    &=
    \int d^kx\bTheta\left(\bx\right) f_{\bX}\left(\bx\right) 
    .
\end{align}

%===============================================================================
\section{Characteristic/moment generating function}

To prove the central limit theorem, first we introduce 
the Fourier transform (or \textbf{characteristic function}) of
a probability distribution. 
Consider a random vector $\bX$ with probability distribution 
$f_{\bX}\left(\bx\right)$, that is $\bX\sim f_{\bX}$. 
We denote the moment generating function with $\psi_{\bX}$, which is
\begin{align}
    \label{eq:moment-generating-function-def}
    \psi_{\bX}\left(\bt\right)
    \equiv
    \int_{-\infty}^{\infty} d^kx e^{i\bt^T\bx} f_{\bX}\left(\bx\right)
    = 
    \mathbb{E}\left[e^{i\bt^T \bX}\right] 
    .
\end{align}
If we set $\bt = -i \tilde{\bt}$, then $\psi_{\bX}$ is called the
\textbf{moment generating function}.
For most probability distributions, there is no meaningful difference between
using $\bt$ or $-i\tilde{\bt}$ (there could potentially only be a difference 
if the distribution had complex poles or branch cuts).
Notice that
\begin{align}
    \mathbb{E}\left[X_{i_1}\cdots X_{i_l}\right]
    =
    \frac{1}{i^l}
    \nabla_{t_{i_1}}\cdots\nabla_{t_{i_l}}\psi_{\bX}\left(\bt\right)
    .
\end{align}
That is, we can obtain the moments of the probability distribution from the
characteristic function (although we need to divide by $1/i^l$).

Perhaps most importantly, notice that since the characteristic function
for a probability distribution is the Fourier transform of the probability density,
we can uniquely map a probability density to its characteristic function and back.
That is, given a characteristic function, we can find the unique probability
density that it corresponds to.

Consider a linear transformation of the random variable $\bX$,
which we call $\bY = A\bX + \bb$. We also call $\by = A\bx + \bb$ (here $A$ is a matrix).
Note that the probability distribution with the volume element remains unchanged, that is 
$d^ky f_{\bY}\left(\by\right)
    =
    d^kx f_{\bX}\left(\bx\right)
    $.
We conclude that
\begin{align}
    \psi_{\bY}\left(\bt\right)
    =&
    \int_{-\infty}^{\infty} d^ky e^{i\bt^T\by} f_{\bY}\left(\by\right)
    \nonumber\\
    =&
    e^{i\bt^T\bb}\int_{-\infty}^{\infty} d^kx e^{i\bt^TA\bx} f_{\bX}\left(\bx\right)
    \nonumber\\
    =&
    e^{i\bt^T\bb}\psi_{\bX}\left(A^T\bt\right)
    .
\end{align}
The last line follows from $\bt^TA\bx = \left(A\bt\right)^T\bx$.
The characteristic function of a sum of independent variables is the product of the characteristic function for each variable.
Define $\bY = \sum_{i=1}^n \bX_i$, we then have
\begin{align}
    \psi_{\bY}\left(\bt\right)
    =&
    \int_{-\infty}^{\infty} dx^k_1\cdots \int_{-\infty}^{\infty} dx^k_n 
        e^{i\bt^T\sum_i\bx_i} f_{\bY}\left(\bx_1,...,\bx_n\right)  
    \nonumber\\
    =&
    \prod_{i=1}^n \int_{-\infty}^{\infty} dx^k_i e^{i\bt^T\bx_i} f_{\bX_i}\left(\bx_i\right)
    \nonumber\\
    =&
    \prod_{i=1}^n \psi_{\bX_i}\left(\bt\right) 
    .
\end{align}
The second line follows from 
$f_{\bY}\left(\bx_1,...,\bx_n\right)=\prod_i f_{\bX_i}\left(\bx_i\right)$,
as all the variables are independent.


%===============================================================================
\section{Central limit theorem}

Let $\bX_1,...,\bX_n$ be $n$ be independent and identically distributed 
random vectors (of dimension $k$ each), 
and let each variable have mean $\bmu$ and covariance matrix $\bSigma$.
The central limit theorem states that the probability distribution of the
average of these variables, 
\begin{align}
    \bar{\bX}_n
    \equiv
    \frac{1}{n}\sum_{i=1}^n\bX_i
    ,
\end{align}
limits to a multivariate normal distribution with mean $\bmu$ and covariance matrix $\bSigma/n$
as $n\to\infty$.
Note that we made no assumption about the probability distribution for
the $\bX_i$, except that the probability distribution has a finite mean
and variance.
We can write the central limit theorem as
\begin{align}
    \lim_{n\to\infty} \sqrt{n}\bar{\bX}_n
    \sim 
    N_k\left(\bmu,\bSigma\right)
    ,
\end{align}
where $N_k$ is the multivariate normal distribution.
To prove this, we make use of the characteristic function for $\bX_n$, which is
\begin{align}
    \label{eq:central-limit-steps}
    \psi_{\bX_n}\left(\bt\right)
    =&
    \prod_{i=1}^n\psi_{\bX_i}\left(\frac{\bt}{n}\right)
    \nonumber\\
    =&
    \left(
        1 
        + 
        i\frac{1}{n}\bt^T\bmu 
        + 
        i^2\frac{1}{2}\frac{1}{n^2}\bt^T\bSigma\bt 
        + 
        \mathcal{O}\left(\frac{1}{n^3}\right)
    \right)^n
    \nonumber\\
    =&
    \mathrm{exp}\left[
        i\bt^T\bmu
        + 
        i^2\frac{1}{2}\bt^T\tilde{\bSigma}\bt
    \right]
    \left(1 + \mathcal{O}\left(\frac{1}{n^3}\right)\right)
    ,
\end{align}
where $\tilde{\bSigma} = \bSigma/n$.
We used the identity
\begin{align}
    \lim_{n\to\infty}\left(1+\frac{a}{n}\right)^n = e^a
    .
\end{align}
To leading order, the last line of \eqref{eq:central-limit-steps} is the characteristic function for $N_k\left(\bmu,\tilde{\bSigma}\right)$ (see \eqref{eq:characteristic-function-multivariate-normal}).
This concludes the proof.

%===============================================================================
\section{Fisher information and the Bernstein–von Mises theorem}

The \textbf{Fisher information} is the negative expectation of the Hessian
of the log likelihood. In terms of components, we have
\begin{align}
    \label{eq:Fisher-information}
    F_{ij}\left(\btheta\right) 
    \equiv&
    -
    \mathbb{E}_{\theta}\left[
        \nabla_{\theta^i}\nabla_{\theta^j}\ln P\left(\bx|\btheta\right)
        \right]
    \nonumber\\
    =&
    -
    \int d^nx P\left(\bx|\btheta\right)
        \nabla_{\theta^i}\nabla_{\theta^j}\ln P\left(\bx|\btheta\right)
    .
\end{align}
Here we view $P\left(\bx|\btheta\right)$ as the PDF for $\bX$, that has dependence
on the \textbf{hyperparamters} $\btheta$.
For example, we could imagine 
$P = \left(2\pi \sigma\right)^{-1/2} \mathrm{exp}\left[-\frac{1}{2}\left(x-\mu\right)\right]$,
and that $\mu,\sigma$ as the hyperparameters of the model.
We set the dimensionality of $\bx$ to be $n$ and the dimensionality of $\btheta$ to be $k$.
The Fisher information can also be written as the variance of the \textbf{score function}.
The score function is
\begin{align}
    \label{eq:score-def}
    s_i\left(\bx;\btheta\right)
    \equiv
    \nabla_{\theta^i}\ln P\left(\bx;\btheta\right)
    .
\end{align}
The expectation of the score function is zero
\begin{align}
    \mathbb{E}\left[s_i\left(\bx;\btheta\right)\right]
    =&
    \int d^nx P\left(\bx;\btheta\right) 
        \nabla_{\theta^i}\ln P\left(\bx;\btheta\right) 
    \nonumber\\
    =&
    \nabla_{\theta^i}
    \int d^nx  
        P\left(\bx;\btheta\right)
    =
    0
    .
\end{align}
We then have
\begin{align}
    \label{eq:fisher-variance-score}
    F_{ij}\left(\btheta\right)
    =&
    -
    \int d^nx P\left(\bx;\btheta\right)
        \nabla_{\theta^i}\nabla_{\theta^j}\ln P\left(\bx;\btheta\right)
    \nonumber\\
    =&
    \int d^nx\left[
        \frac{1}{P\left(\bx;\btheta\right)}
        \nabla_{\theta^i}P\left(\bx;\btheta\right)
        \nabla_{\theta^j}P\left(\bx;\btheta\right)
        -
        \nabla_{\theta^i}\nabla_{\theta^j}P\left(\bx;\btheta\right)
    \right]
    \nonumber\\
    =&
    \int d^nx
        P\left(\bx;\btheta\right)
        \nabla_{\theta^i}\ln P\left(\bx;\btheta\right)
        \nabla_{\theta^j}\ln P\left(\bx;\btheta\right)
    \nonumber\\
    =&
    \mathbb{E}_{\theta}\left[s_is_j\right]
    \nonumber\\
    =&
    \mathbb{V}_{\theta,ij}\left[
        {\bf s}\left(\bx;\btheta\right)
    \right]
    .
\end{align}
That is, the Fisher information is the variance of the score.

Let $\hat{\btheta}$ be the maximum likelihood estimator for $\btheta$.
Under appropriate regularity conditions,
the likelihood $\mathcal{L}\left(\btheta\right)$ 
tends towards a multivariate Gaussian function with mean $\hat{\btheta}$ and 
covariance matrix given by the inverse Fisher information divided by the number of measurements of the data $n$, 
$\tilde{\bF}^{-1}=\bF^{-1}/n$. 
In equations, we have
\begin{align}
    \lim_{n\to\infty}P\left(\btheta|\bx\right)
    =
    \lim_{n\to\infty}\frac{\pi\left(\btheta\right)\prod_{i=1}^nP\left(\bx_i|\btheta\right)}{\mathcal{Z}\left(\bx\right)}
    =
    N\left(\hat{\btheta},\tilde{\bF}^{-1}\right)
    .
\end{align}
This is known as the \textbf{Bernstein–von Mises theorem} (BvM theorem for short). 
We provide a rough sketch of how the proof goes. 
For more details see \cite{wasserman2010statistics}.
The log likelihood is
\begin{align}
    \ln\mathcal{L}\left(\btheta\right)
    =
    \sum_{i=1}^N\ln P\left(\bx_{i}|\btheta\right)
\end{align}
We Taylor series expand the derivative of the log-likelihood to linear order about a point 
$\btheta_0$
\begin{align}
    \nabla_{\theta^i}\ln\mathcal{L}\left(\btheta\right)
    =&
    \nabla_{\theta^i}\ln\mathcal{L}\left(\btheta\right)\big|_{\btheta=\btheta_0}
    +
    \nabla_{\theta^i}\nabla_{\theta^j}\ln\mathcal{L}\left(\btheta\right)\big|_{\btheta=\btheta_0}
    \left(\theta-\theta_0\right)^j
    +
    \cdots
    .
\end{align}
Setting $\btheta=\hat{\btheta}$, relabeling $\btheta_0\to\btheta$, 
dropping the ``$\big|_{\btheta=\btheta_0}$'' to reduce clutter, and rearranging gives us 
\begin{align}
    \sqrt{n}\left(\hat{\theta}^i - \theta^i\right)
    =&
    -
    \left(\frac{1}{n}\nabla_{\btheta}\nabla_{\btheta}\ln\mathcal{L}\left(\btheta\right)\right)^{-1}_{ij}
    \left(\frac{1}{\sqrt{n}}\nabla_{\theta^j}\ln\mathcal{L}\left(\btheta\right)\right)
    .
\end{align}
We used that at the maximum likelihood estimator, $\hat{\btheta}$, the derivative of the likelihood is zero.
As $n\to\infty$, we see that
\begin{align}
    \lim_{n\to\infty}\nabla_{\theta^j}\ln\mathcal{L}\left(\btheta\right)
    =&
    \frac{1}{\sqrt{n}}\lim_{n\to\infty}\sum_{m=0}^n\nabla_{\theta^j}\ln P\left(\btheta|\bx_{m}\right)
    \nonumber\\
    \to & \sim
    N_k\left({\bm 0},\bF\right)
    .
\end{align}
The above result followed from the central limit theorem: the mean of the score is zero, 
and the variance of the score is the Fisher information.
By the law of large numbers we can average
\begin{align}
    \lim_{n\to\infty}
    \frac{1}{n}\left(-\left(\nabla_{\btheta}\nabla_{\btheta}\ln\mathcal{L}\left(\btheta\right)\right)_{ij}\right)
    =&
    \lim_{n\to\infty}
    \frac{1}{n}\sum_{m=1}^n
    \left(-\nabla_{\theta^i}\nabla_{\theta^j}\ln P\left(\btheta|\bx_{m}\right)\right)
    \nonumber\\
    \to&
    F_{ij}
    .
\end{align}
Thus the variance of the limit is modified to be $\bF^{-1}\bF\bF^{-1}=\bF^{-1}$.
We can then conclude that
\begin{align}
    \lim_{n\to\infty}\sqrt{n}\left(\hat{\btheta} - \btheta\right)
    \sim
    N_k\left({\bm 0},\bF^{-1}\right)
    .
\end{align}
Or in other words
\begin{align}
    \lim_{n\to\infty}\sqrt{n}\hat{\btheta}
    \sim
    N_k\left(\hat{\btheta},\bF^{-1}\right)
    .
\end{align}
We have not been careful by what we mean by ``$\to$'' and ``$\sim$'' here--in fact there
are different notions of convergence that go into the full proof
(see for example \cite{wasserman2010statistics}).

We can understand the BvM theorem heuristically as follows.
As we collect more data, the posterior probability becomes increasingly ``peaked'' near
the maximum likelihood estimator.
We can then Taylor series about the maximum of the log-likelihood to quadratic order.
Exponentiating the log-likelihood gives us a multivariate normal with the 
inverse Fisher information as the covariance matrix.

%===============================================================================
\section{Fisher information and the Cram\'{e}r-Rao bound}

Consider an estimator $\bTheta$ for model parameters $\btheta$.
Let $\bSigma$ is the covariance matrix for the estimator $\bTheta$,
let $\mathbb{E}\left[\bTheta\right]=\bpsi$,
and let $\bF$ be the Fisher information evaluated at $\bpsi$.
The Cram\'{e}r-Rao bound states that
\begin{align}
    \label{eq:cramer-rao-biased}
    \Sigma_{ij} 
    \geq 
    \nabla_{\theta_m}\psi_i
    \nabla_{\theta_n}\psi_j
    F_{mn}^{-1} 
    ,
\end{align}
If $\bTheta$ is an unbiased estimator ($\bpsi=\btheta$), then
\eqref{eq:cramer-rao-biased} reduces to 
\begin{align}
    \label{eq:cramer-rao-unbiased}
    \Sigma_{ij} 
    \geq 
    F_{ij}^{-1} 
    ,
\end{align}
If $\bTheta$ is a biased estimator, then 
$\nabla_{\theta_i}\psi_j = \delta_{ij} + \nabla_{\theta_i}b_j$, 
where $b_j$ is the bias.
The Cram\'{e}r-Rao bound can be used to interpret the Fisher matrix as an estimate
for the lowest error one could achieve for an unbiased estimator.
For biased estimators though, we see that the Fisher information does not give a lower
bound on the elements of the covariance matrix, since it is possible that the bias
could be negative, $\nabla_{\theta_i}b_j<0$.
That is, biased estimators can have \emph{smaller} 
covariance matrix elements than unbiased estimators.
If the error from the bias is less than the error from the covariance 
(for example, if one only has a few measurements of noisy data),
a biased estimator can sometimes be superior to an unbiased
estimator in determining $\btheta$.

Here we provide the outline of a proof of \eqref{eq:cramer-rao-biased}.
First we prove a generalization of the Cauchy-Schwartz inequality.
Let $\by$ and $\bz$ be random vectors (not necessarily of the same dimensionality).
Then
\begin{align}
    \label{eq:generalized-Cauchy-Schwartz}
    \mathbb{V}_{ij}\left[\bz\right]
    \geq
    \mathbb{C}_{ip}\left[\bz,\by\right]
    \mathbb{V}_{pq}\left[\by\right]^{-1}
    \mathbb{C}_{qj}\left[\by,\bz\right]
    .
\end{align}
To prove this, we define $\bu \equiv \by - \bmu_{\by}$
and $\bv \equiv \bz - \bmu_{\bz}$, so $\bmu_{\bu}=0$ and $\bmu_{\bv}=0$.
For any matrix $\bA$ we have the following matrix inequality 
(we insert the matrix in case $\bv$ and $\bu$ have different dimensionality)
\begin{align}
    \left(\bv + \bA\bu\right)
    \left(\bv + \bA\bu\right)^T
    \geq 
    0
    .
\end{align}
Taking the expectation of this and expanding, we have
\begin{align}
    \mathbb{E}\left[\bv\bv^T\right]
    +
    \bA\mathbb{E}\left[\bu\bv^T\right]
    +
    \mathbb{E}\left[\bv\bu^T\right]\bA^T
    +
    \bA\mathbb{E}\left[\bu\bu^T\right]\bA^T
    \geq
    0
    .
\end{align}
Set $\bA = - \mathbb{E}\left[\bu\bv^T\right] \mathbb{E}\left[\bu\bu^T\right]^{-1}$.
The last two terms cancel, and we are left with
\begin{align}
    \mathbb{E}\left[\bv\bv^T\right]
    \geq  
    \mathbb{E}\left[\bu\bv^T\right] 
    \mathbb{E}\left[\bu\bu^T\right]^{-1} 
    \mathbb{E}\left[\bu\bv^T\right]
    .
\end{align}
Re-introducing $\by$ and $\bz$, and using the definition of the covariance 
\eqref{eq:variance-def} and variance \eqref{eq:covariance-def}, 
we have \eqref{eq:generalized-Cauchy-Schwartz},
\begin{align}
    \mathbb{V}_{ij}\left[\bz\right]
    \geq  
    \mathbb{C}_{ip}\left[\by,\bz\right] 
    \mathbb{V}_{pq}\left[\by\right]^{-1} 
    \mathbb{C}_{qj}\left[\by,\bz\right]
    .
\end{align}
This completes the proof of the generalized Cauchy-Schwartz inequality.

We now prove \eqref{eq:cramer-rao-biased}.
We use \eqref{eq:generalized-Cauchy-Schwartz}, and set
\begin{align}
    \bz
    =
    \bTheta
    ,\qquad
    \by
    =
    \bs
    ,
\end{align}
where $\bTheta$ is an estimator for $\btheta$,
and $\bs$ is the score (see \eqref{eq:score-def}). 
The covariance between $\bTheta$ and $\bs$ is
\begin{align}
    \mathbb{C}_{ij}\left[\bTheta,\bs\right]
    =&
    \mathbb{E}\left[
        \left(\Theta_i-\mu_{\Theta_i}\right)
        \left(s_i-\mu_{s_j}\right)
    \right]
    \nonumber\\
    =&
    \mathbb{E}\left[\Theta_is_j\right]
    \nonumber\\
    =&
    \int d^nx P\left(\bx|\btheta\right) \Theta_i \nabla_{\theta_j} \ln P\left(\bx|\btheta\right)
    \nonumber\\
    =&
    \nabla_{\theta_j}\mathbb{E}\left[\Theta_i\right]
    \nonumber\\
    =&
    \nabla_{\theta_j}\psi_i
    .
\end{align}
We also have 
\begin{align}
    \mathbb{V}_{ij}\left[\bTheta\right]
    =
    \Sigma_{ij}
    ,\qquad
    \mathbb{V}_{ij}\left[\bs\right]
    =
    F_{ij}
    .
\end{align}
We have defined $\bSigma$ to be the covariance matrix of $\bTheta$, 
and used that Fisher information is the variance of the score 
(see \eqref{eq:fisher-variance-score}),
Plugging this all into \eqref{eq:generalized-Cauchy-Schwartz}, 
we obtain the Cram\'{e}r-Rao bound
\begin{align}
    \Sigma_{ij}\Big|_{\btheta=\bmu_{\bTheta}} 
    \geq 
    \nabla_{\theta_p}\psi_i
    \nabla_{\theta_1}\psi_j
    F_{pq}^{-1}\left(\btheta\right) 
    .
\end{align}
