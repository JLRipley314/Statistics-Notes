We consider the problem of measuring/extracting a signal from a noisy data timestream.
Calling the data $x(t)$, we then want to find a signal $s(t)$ given noise $n(t)$, where
\begin{align}
    \label{eq:x-signal-n}
    x\left(t\right) = s\left(t\right) + n\left(t\right)
    .
\end{align}
We assume $x,s,n$ are all real or complex scalarfunctions.
The noise $n(t)$ can be modeled as a \textbf{stochastic process} (which implies that $x(t)$ is a stochastic process)
Our main goal is to derive the likelihood function for a time series of the form
\eqref{eq:x-signal-n} when $n$ takes the form of colored stationary noise, and
to derive the matched filtering theorem.

%===============================================================================
\section{Basic definitions}

A function $n(t)$ is a \textbf{stochastic process}, if $n(t)$ is a random variable at each time $t$.
That is, at each time $t$ we essentially draw $n(t)$ from a probability distribution.
This probability distribution may depend on $t$, and the previous history of values of $x$, for example.
If $t$ is a discrete variable, and the probability distribution for $n(t_i)$ depends on $n(t_{i-1})$, then $n(t_i)$ is a \textbf{Markov chain}.
If the probability distribution for $n(t)$ is independent of $t$, then $n(t)$ is a \textbf{stationary process}.

%===============================================================================
\section{Correlation and covariance}

We denote the mean and variance of a time series $x(t)$ with $\mu_x$ and $\sigma_x$, respectively. 
We define the \textbf{covariance} between two stochastic processes $x_1$ and $x_2$ at times $t_1$ and $t_2$ to be 
\begin{align}
    C_{x_1,x_2}\left(t_1,t_2\right)
    \equiv
    \mathbb{E}\left[
        \left(x_1\left(t_1\right) - \mu_{x_1\left(t_1\right)}\right)
        \left(x^*_2\left(t_2\right) - \mu^*_{x_2\left(t_2\right)}\right)
    \right]
    .
\end{align}
We define the \textbf{autocovariance} for a stocahstic process to be 
\begin{align}
    K_{x}\left(t_1,t_2\right)
    \equiv
    C_{xx}\left(t_1,t_2\right)
    .
\end{align}
We define the \textbf{correlation} for a stochastic process to be 
\begin{align}
    R_{x_1,x_2}\left(t_1,t_2\right)
    \equiv
    \mathbb{E}\left[x_1\left(t_1\right)x_2^*\left(t_2\right)\right]
    =
    K_{x} + \mu_x^2
    .
\end{align}
We define the \textbf{autocorrelation} for a stochastic process to be
\begin{align}
    R_{x}\left(t_1,t_2\right)
    \equiv
    \mathbb{E}\left[x\left(t_1\right)x^*\left(t_2\right)\right]
    =
    K_{x} + \mu_x^2
    .
\end{align}

We define the \textbf{energy} of a time series $s$ to be 
\begin{align}
    E_{x}
    \equiv
    \int_{-\infty}^{\infty}dt \left|x\left(t\right)\right|^2
    =
    \int_{-\infty}^{\infty}df \left|\tilde{x}\left(f\right)\right|^2
    .
\end{align}
The last expression follows from Parseval's theorem.
We define the \textbf{energy spectral density} to be
\begin{align}
    \hat{S}_{x}\left(f\right)
    \equiv
    \left|\tilde{x}\left(f\right)\right|^2
    .
\end{align}

%===============================================================================
\section{Stationary and weak-sense stationary stochastic processes}

A stochastic process is said to be \textbf{stationary} (or \textbf{strict-sense stationary}) if its joint probability distribution does not change under time shifts. 
That is, 
\begin{align}
    f_{X\left(t_1\right)}\left(x\right)
    =
    f_{X\left(t_1+\Delta t\right)}\left(x\right)
    \qquad
    \forall \Delta t
    .
\end{align}
Said another way, $X\left(t+\Delta t\right) \sim X\left(t\right)$.

A stochastic process is said to be \textbf{weak-sense stationary (WSS)} (or \textbf{wide-sense stationary}) if 
\begin{align}
    \mathbb{E}\left[X\left(t_1\right)\right]
    =&
    \mathbb{E}\left[X\left(t_2\right)\right]
    &
    \forall t_1,t_2
    ,\\
    \mathbb{E}\left[X\left(t_1\right)X\left(t_2\right)\right]
    =&
    \mathbb{E}\left[X\left(t_3\right)X\left(t_4\right)\right]
    &
    \forall t_1,t_2, t_3, t_4
    .
\end{align}

That is, for a WSS process we have
\begin{align}
    \label{eq:WSS-mean}
    \mathbb{E}\left[X\left(t_1\right)\right]
    =&
    \mu_X
    ,\\
    \label{eq:WSS-autocorrelation}
    R_x\left(\tau\right) 
    =&
    \mathbb{E}\left[x\left(t+\tau\right)x^*\left(t\right)\right]
    ,
\end{align}
where $\mu_X$ is independent of time.
For a WSS/stationary process, we can write the expectation of $x$ at a given instant as the average of $x$ over all time 
\begin{align}
    \mathbb{E}\left[x\right]
    =
    \lim_{T\to\infty}\frac{1}{T}\int_{-\infty}^{\infty}dt x_T\left(t\right)
    ,
\end{align}
and similarly for functions of $x(t)$.
Here $x_T\left(t\right)$ is defined to be
\begin{align}
    x_T\left(t\right) 
    \equiv& 
    w_T\left(t\right)x\left(t\right)
    ,\\
    w_T\left(t\right)
    \equiv&
    \begin{cases}
        1 & |t|<T/2
        \\
        0 & \mathrm{otherwise}
    \end{cases}
    .
\end{align}
In other words, we can write
\begin{align}
    \label{eq:autocorrelation-WSS-integral}
    R_x\left(\tau\right)
    =
    \lim_{T\to\infty}\frac{1}{T}\int_{-\infty}^{\infty}dt 
        x_T\left(t+\tau\right)x_T^*\left(t\right)
    .
\end{align}
We emphasize that for WSS processes we can replace ensemble averages with time averages.
This is extremely useful in practice, as we can then determine the statistical properties
of WSS by taking repeated time measurements of observables of the time series.
While it is common to assume a given time stream is WSS, 
most real world data is at best approximately WSS, typically over a short time scale.

Stationary stochastic processes have support over the entire real line, 
so the energy integrals defined above typically diverge. 
For these processes, instead one looks at the \textbf{power}
\begin{align}
    P_{x}
    \equiv&
    \lim_{T\to\infty}\frac{1}{T}\int_{-\infty}^{\infty}dt \left|x_T\left(t\right)\right|^2
    \nonumber\\
    =&
    \lim_{T\to\infty}\frac{1}{T}\int_{-\infty}^{\infty}df \left|\tilde{x}_T\left(f\right)\right|^2
    \nonumber\\
    =&
    \int_{-\infty}^{\infty}df S_x\left(f\right) 
    .
\end{align}
One the last line we have used the \textbf{power spectral density}, 
which is defined to be
\begin{align}
    \label{eq:power-spectral-density}
    S_{x}\left(f\right)
    \equiv
    \lim_{T\to\infty}\frac{1}{T}\left|\tilde{x}_T\left(f\right)\right|^2
    .
\end{align}
If $x(t)$ is real, then $\tilde{x}_T\left(-f\right)=\tilde{x}^*_T\left(f\right)$, and
it is common to define the power spectral density to be
\begin{align}
    S_x\left(f\right)
    \equiv
    \lim_{T\to\infty}\frac{2}{T}\left|\tilde{x}_T\left(f\right)\right|^2
    ,
\end{align}
and to write
\begin{align}
    P_x
    =
    \int_0^{\infty}df S_x\left(f\right)
    .
\end{align}

Finally, we consider the expectation value of the Fourier transform
of a stationary signal
\begin{align}
    \mathbb{E}\left[\tilde{x}\left(f^{\prime}\right)\tilde{x}^*\left(f\right)\right]
    =&
    \int_{-\infty}^{\infty}dt \int_{-\infty}^{\infty}dt^{\prime}
        e^{- 2\pi i \left(f t - f^{\prime}t^{\prime}\right)}
        \mathbb{E}\left[x\left(t\right)x\left(t^{\prime}\right)\right]
    \nonumber\\
    =&
    \int_{-\infty}^{\infty}dt e^{-2\pi i \left(f-f^{\prime}\right)t}
    \int_{-\infty}^{\infty}d\tau e^{-2\pi i f^{\prime}\tau}
        \mathbb{E}\left[x\left(t+\tau\right)x\left(t\right)\right]
    \nonumber\\
    =&
    \int_{-\infty}^{\infty}dt e^{-2\pi i \left(f-f^{\prime}\right)t}
        \tilde{R}_x\left(f\right) 
    \nonumber\\
    =&
    \delta\left(f-f^{\prime}\right)S_x\left(f\right) 
    .
\end{align}
On the third line we used that $x(t)$ was stationary.

%=====================================================================================
\section{Wiener-Khinchin theorem}

The power spectral density and the Fourier transform of the autocorrelation are
equal for WSS processes. 
This is known as the \textbf{Wiener–Khinchin theorem}.
To prove this, we set $\tau\equiv t_1-t_2$. We then write 
\begin{align}
    \tilde{R}_{x}\left(f\right)
    =& 
    \int_{-\infty}^{\infty}d\tau e^{-2\pi i f \tau}R_{x}\left(\tau\right)
    \nonumber\\
    =& 
    \lim_{T\to\infty}\frac{1}{T}\int_{-\infty}^{\infty}d\tau e^{-2\pi i f \tau} 
    \int_{-\infty}^{\infty} dt x_T\left(t+\tau\right)x^*_T\left(t\right)
    \nonumber\\
    =& 
    \int_{-\infty}^{\infty}df^{\prime}  
    \int_{-\infty}^{\infty}d\tau
        e^{2\pi i \left(f^{\prime}-f\right)\tau} 
        \lim_{T\to\infty}\frac{1}{T}\left|\tilde{x}_T\left(f^{\prime}\right)\right|^2 
    \nonumber\\
    =& 
    S_{x}\left(f\right) 
    .
\end{align}
On the second line we used the formula for the autocorrelation function
for WSS processes \eqref{eq:autocorrelation-WSS-integral}.
On the third line we used the convolution theorem for Fourier transforms.
On the last line we used that $\int d\tau e^{i\tau f}=\delta(f)$, and
used the definition of $S_x(f)$ \eqref{eq:power-spectral-density}.
%===============================================================================
\section{Gaussian white noise}

As a special case of a stationary stochastic process, 
we first consider \textbf{Gaussian white noise}. 
By \textbf{Gaussian}, we mean that the probability distribution for 
$x\left(t\right)$ for each $t_i$ is a Gaussian 
\begin{align}
    x\left(t_i\right) \sim \mathcal{N}\left(\mu_i,\sigma_i^2\right)
    .
\end{align}
By \textbf{white}, we mean that the $x\left(t_i\right)$ are uncorrelated, 
and that the means $\mu_i=0$.
The autocorrelation function then is 
\begin{align}
    \label{eq:autocorrelation-white-noise}
    R_{x}\left(t_i,t_j\right)
    =
    \sigma^2 \delta_{ij}
    .
\end{align}
Notice that $\sigma^2$ does not depend on $t_i$. 
We see that white noise is stationary.
The autocorrelation function can be written in terms of $\tau \equiv t_i-t_j$ as
\begin{align}
    \label{eq:autocorrelation-white-noise}
    R_{x}\left(\tau\right) = \sigma^2 \delta\left(\tau\right) 
    .
\end{align}
By the Wiener-Kinchin theorem, we can compute the power spectral density from
the Fourier transform of the autocorrelation function
\begin{align}
    S_{x}\left(f\right)
    =&
    \int_{-\infty}^{\infty} dt e^{2\pi i f t}R_{x}\left(t\right)
    \nonumber\\
    =&
    \sigma^2
    .
\end{align}
We see for Gaussian white noise, the power spectral density is a constant--there
is constant power across all frequencies.
For real functions, the integral over frequencies goes from $[0,\infty)$,
and we define
\begin{align}
    S_x\left(f\right)
    =
    2\sigma^2 
    .
\end{align}

%===============================================================================
\section{Likelihood function for a series of measurements with colored stationary noise
\label{sec:likelihood-function-colored-noise}}

Our treatment roughly follows \cite{Creighton:2011zz} (see also \cite{Finn:1992wt}).
We consider a series of a continuous time stream of observations $y\left(t\right)$.
We assume that $y\left(t\right)$ can be related to a convolution of
a time stream drawn from Gaussian white noise
\begin{align}
    y\left(t\right)
    =
    \int_{-\infty}^{\infty}dt^{\prime}\gamma\left(t-t^{\prime}\right)x\left(t^{\prime}\right)
    .
\end{align}
Here $\gamma$ is the kernel and $x$ is a time stream drawn from a Gaussian
distribution with constant.
We assume $y,x,\gamma$ are all real functions.
The Fourier transform gives us
\begin{align}
    \tilde{y}\left(f\right)
    =
    \tilde{\gamma}\left(f\right)\tilde{x}\left(f\right)
    .
\end{align}
The power spectral density of $y$ then is
\begin{align}
    S_{y}\left(f\right)
    =
    \left|\tilde{\gamma}\left(f\right)\right|^2S_x\left(f\right)
    =
    \left|\tilde{\gamma}\left(f\right)\right|^2\sigma
    .
\end{align}
The main point of adding the convolution is that we can consider processes
with \textbf{colored noise}, that noise where the power spectral density
can vary with frequency. We can do this by choosing some $\sigma$, and
then choosing a $\gamma$ such that $|\tilde{\gamma}\left(f\right)|^2$
gives us the spectral density we desire.

We note that $y\left(t\right)$ describes a WSS process, as
\begin{align}
    \mathbb{E}\left[y\left(t_1\right)y\left(t_2\right)\right]
    =&
    \int_{-\infty}^{\infty}dt_1^{\prime}
    \int_{-\infty}^{\infty}dt_2^{\prime}
        \gamma\left(t_1-t_1^{\prime}\right)
        \gamma\left(t_2-t_2^{\prime}\right)
        \mathbb{E}\left[x\left(t_1^{\prime}\right)x\left(t_2^{\prime}\right)\right]
    \nonumber\\
    =&
    \sigma^2
    \int_{-\infty}^{\infty}dt^{\prime}
        \gamma\left(t_1-t^{\prime}\right)
        \gamma\left(t_2-t^{\prime}\right)
    \nonumber\\
    =&
    \sigma^2
    \int_{-\infty}^{\infty}dt
        \gamma\left(\tau-t\right)
        \gamma\left(t\right)
    .
\end{align}
On the second line we used ~\eqref{eq:autocorrelation-white-noise},
while on the third we set $t=t_2-t^{\prime}$.

We consider a discretized set of $N$ points (evenly spaced) from
$x(t)$ and $y(t)$. We write the vectors 
${\bf x}$ and ${\bf y}$ where the componets are, e.g.
${\bf x}_i \equiv x(t_i)$.
We define the discretized matrix $\Gamma_{ij} \equiv \gamma\left(t_i-t_j\right)$.
We then have
\begin{align}
    y_i = \Gamma_{ij} x_j
\end{align}
We define the matrix
\begin{align}
    \Sigma_{ij}
    \equiv 
    \frac{\sigma^2}{\Delta t} \delta_{ij}
    ,
\end{align}
where $\Delta t \equiv T/\left(N-1\right)$, and $N$ is the number of discretized points. 
We choose this scaling so that the autocorrelation of $x_i$ approaches the
correct behavior in the continuum limit, as we show below.
The probability distribution for each ${\bf x}$ is
\begin{align}
    P_{\bf X}\left({\bf x}\right)
    =
    \left(\frac{1}{\sqrt{2\pi}}\right)^N\frac{1}{\sqrt{\det{\Sigma}}}
    \exp\left[-\frac{1}{2}x_i\Sigma^{-1}_{ij}x_j\right]
    .
\end{align}
The correlation for $x$ is then
\begin{align}
    \mathbb{E}\left[x_ix_j\right]
    =&
    \int d^Nx P_{\bf X}\left({\bf x}\right) x_i x_j
    \nonumber\\
    =&
    \int d^Nx 
        \left(\frac{1}{\sqrt{2\pi}}\right)^N\frac{1}{\sqrt{\det{\Sigma}}}
        \exp\left[-\frac{1}{2}x_i\Sigma^{-1}_{ij}x_j\right]
        x_i x_j
    \nonumber\\
    =&
    \sigma^2\frac{\delta_{ij}}{\Delta t}
    .
\end{align}
In the limit $\Delta t\to0$, this approaches $\sigma^2\delta\left(t_i-t_j\right)$,which is the autocorrelation function for Gaussian white noise; 
see ~\eqref{eq:autocorrelation-white-noise}.

We obtain the probability distribution for ${\bf y}$ under a linear transformation
of variables
(note the ordering of the indices)
\begin{align}
    P_{\bf Y}\left({\bf y}\right)
    =
    \left(\frac{1}{\sqrt{2\pi}}\right)^N
    \frac{1}{\sqrt{\det{\Sigma}\det{\Gamma}}}
    \exp\left[-\frac{1}{2}y_i\Gamma_{mi}^{-1}\Sigma^{-1}_{mn}\Gamma^{-1}_{nj}y_j\right]
    .
\end{align}
This expression gives the probability the $N$ draws.
We now need to take the continuum limit.
First we look at the argument of the exponential  
\begin{align}
    y_i\Gamma^{-1}m_{mi}\Sigma^{-1}_{mn}\Gamma^{-1}_{nj}y_j
    =&
    \frac{1}{\sigma^2}x_ix_i \Delta t
    \nonumber\\
    \to&
    \frac{2}{S_x} \int_{t_s}^{t_f}dt \left|x\left(t\right)\right|^2
    \nonumber\\
    \approx&
    \frac{2}{S_x} \int_{-\infty}^{\infty}dt \left|x\left(t\right)\right|^2
    \nonumber\\
    =&
    \frac{4}{S_x} \int_0^{\infty}df \left|\tilde{x}\left(f\right)\right|^2
    \nonumber\\
    =&
    4\int_{-\infty}^{\infty}df 
        \frac{\left|\tilde{y}\left(f\right)\right|^2}{S_y}
    .
\end{align}
Here $t_s,t_f$ are the start and end times for the series $x(t)$.
On the first line we used $x_i = \Gamma^{-1}_{ij}x_j$, and that
$\Sigma_{ij}^{-1} = \sigma^{-1}\delta_{ij}$.
On the second and third lines we converted the Riemann sum to an
integral (we took the continuum limit).
We approximated the start/end times with $\pm\infty$.
On the last line we use $\tilde{x}=\tilde{y}/\tilde{\gamma}$, 
$S_y= \left|\tilde{\gamma}\left(f\right)\right|^2S_x$, and that
$S_x$ is a constant, so we can pull it into the integral.
Remember that we assume that $x,y,\gamma$ are all real, so that for example
$x\left(-f\right)=x^*\left(-f\right)$.
Ignoring the constant normalization factor, we see that the probability density
function (the likelihood function) for $y\left(t\right)$ is
\begin{align}
    \label{eq:likelihood-function-colored-noise}
    P\left(y\left(t\right)\right)
    \propto
    \exp\left[-\frac{1}{2}\left(y,y\right)\right]
    ,
\end{align}
where we have defined the inner product
\begin{align}
    \label{eq:inner-product-matched-filter}
    \left(a,b\right)
    \equiv
    2 \int_0^{\infty}df
        \frac{
            a\left(f\right)b^*\left(f\right)
            +
            b\left(f\right)a^*\left(f\right)
        }{
            S_y\left(f\right)
        }
        .
\end{align}
Here $S_y\left(f\right)$ is the spectral noise density for the process,
and $a,b$ can represent the Fourier transform of particular draws.
We interpret ~\eqref{eq:likelihood-function-colored-noise} as the likelihood
function (up to normalization) for colored WSS noise.
We call ~\eqref{eq:inner-product-matched-filter} a \textbf{matched filter}.
We define the \textbf{signal to noise ratio (SNR)} for a signal $s$ with noise $n$ to be 
\begin{align}
    \rho^2
    \equiv
    \left(s,s\right)
    =
    4 \int_0^{\infty}df
        \frac{
            \left|s\left(f\right)\right|^2
        }{
            S_n\left(f\right)
        }
    .
\end{align}

%===============================================================================
\section{Matched filter theorem}

We next derive the optimal test statistic for extracting a singal from
WSS colored noise.
We consider a time series $x(t)$ that can be written as
\begin{align}
    x\left(t\right) = s\left(t\right) + n\left(t\right)
    .
\end{align}
We assume that $n(t)$ can be written as a convolution with Gaussian
white noise, as we described in Sec.~\eqref{sec:likelihood-function-colored-noise}.
We assume we are searching for a signal $s\left(t\right)$ that we know how
to compute. 
We these assumptions, we can compute the optimal test statistic to 
distinguish between the two following hypothesis:
\begin{enumerate}
    \item[] Null hypothesis $\mathcal{H}_0$: $x\left(t\right) = n\left(t\right)$.
    \item[] Alternative hypothesis $\mathcal{H}_1$: $x\left(t\right) = s_1\left(t\right) + n\left(t\right)$.
\end{enumerate}
Here $s_1\left(t\right)$ is a signal we are guessing is in the data.
We compare the two hypothesis by computing the likelihood ratio
(likelihood for short for the rest of this section)
\begin{align}
    \Lambda\left(\mathcal{H}_1|x\right)
    \equiv
    \frac{P\left(x|\mathcal{H}_1\right)}{P\left(x|\mathcal{H}_0\right)}
    .
\end{align}
We use the likelihood function ~\eqref{eq:likelihood-function-colored-noise}.
We next show that $s_1\propto s$ maximizes $\Lambda$, which is the \textbf{matched filtering theorem}.

If the null hypothesis is true, then the probability density function goes as
\begin{align}
    P\left(x|\mathcal{H}_0\right)
    \propto
    \exp\left[-\frac{1}{2}\left(x,x\right)\right]
    .
\end{align}
If the alternative hypothesis is true, then the probability density function
goes as 
\begin{align}
    P\left(x|\mathcal{H}_0\right)
    \propto
    \exp\left[-\frac{1}{2}\left(x-s_1,x-s_1\right)\right]
    .
\end{align}
We have used ~\eqref{eq:inner-product-matched-filter}, with the noise power spectral
density given by $S_n(f)$.
The normalization factors cancel out in the likelihood ratio, and we are left with
\begin{align}
    \Lambda\left(\mathcal{H}_1|x\right)
    =&
    \mathrm{exp}\left[
            - 
            \frac{1}{2} \left(x-s_1,x-s_1\right) + \frac{1}{2}\left(x,x\right)
        \right]
    \nonumber\\
    =&
    \exp\left[
             \left(x,s_1\right) - \frac{1}{2}\left(s_1,s_1\right) 
        \right]
    .
\end{align}

The matched filtering theorem states that the likelihood ratio 
$\Lambda\left(\mathcal{H}_1|x\right)$ is maximized when $s_1\propto s$.
To show this, we first note that likelihood ratio is maximized when the log-likelihood
ratio $L$ is maximized. The log-likelihood is
\begin{align}
    L\left(s_1\right)
    \equiv
    \left(n,s_1\right)
    +
    \left(s,s_1\right)
    -
    \frac{1}{2}\left(s_1,s_1\right)
    .
\end{align}
We only consider $s_1$ such that $\left(n,s_1\right)=0$
(this also holds for the ``true'' signal $s$).
Moreover, we fix $\left(s_1,s_1\right)=c_1$, where $c_1$ is a constant
(otherwise the likelihood could be arbitrarily big or small by rescaling the amplitude of $s_1$)
Maximizing the likelihood then reduces to maximizing
\begin{align}
    L\left(s_1\right)
    = 
    \left(s,s_1\right)
    -
    \frac{1}{2}c_1
    .
\end{align}
By the Cauchy-Schwartz inequality, we have that
\begin{align}
    \left(s,s_1\right)
    \leq 
    \sqrt{\left(s,s\right)}
    \sqrt{\left(s_1,s_1\right)}
    .
\end{align}
Equality only holds when $s_1\propto s$.
We conclude that choosing  $s_1\propto s$ 
maximizes the likelihood function (up to a proportionaly constant,
which is fixed by the condition $(s_1,s_1)=c_1$). 

The task of finding a signal in colored WSS noise then reduces to
finding a filtering function $s_1$ that is orthogonal to the noise,
and that maximizes the value of the matched filter $\left(x,s_1\right)$.
In practice, we can determine the noise profile of the detector by
measuring the response of the detector in the (assumed) absence of any signal.

The matched filtering theorem is powerful, but it relies on several strong assumption
that are only approximately met in practice. 
First, it assumes that we know what we are looking for--that is, that we have a 
\textbf{template bank} of templates $s_i(t)$ that we can convolve with the data.
Even if we do have a template bank, it can be very computationally expensive to
search for the $s_i$ that fits the data best, especially if the parameter space
for $s_i$ is large. 
Efficiently evaluating the likelihood and searching through parameter space remains
a topic of active research in, e.g. the gravitational wave astronomy community
(for a review, see e.g. \cite{Creighton:2011zz}).
The matched filtering theorem also assumes the noise is stationary or WSS. 
Most kinds of detectors (say a phone line, or a gravitational wave detector) suffer
from non-stationary noise, often called \textbf{glitches}.
Provided those are well enough understood, they can be subtracted out of the signal,
although in practice it can be difficult to completely remove glitches from a time stream.
