Parts of this chapter are inspired by sections in \cite{zhou2008practical,feller1968introduction}.

I'll assume familiarity with Chptr.~\ref{chap:time-series-analysis} and Chptr.~\ref{chap:markov-chains}.
A \textbf{martingale} is a sequence of random vectors (a stochastic process) $\left\{\btheta_1,\btheta_2,...\right\}$, where the expectation of step $k+1$ is equal to the value of the chain at step $k$.
That is
\begin{align}
    \label{eq:martingale-property}
    \mathbb{E}\left[\btheta_{k+1}|\btheta_{k}=\bi,\btheta_{k-1}=\bi_{k-1},\cdots,\btheta_0=\bi_0\right]
    =
    \bi
    .
\end{align}

Notice that the martingale property \eqref{eq:martingale-property} implies that the (unconditioned) expectations at every step are equal to one another.
We assume a probabilistic distribution for the initial data $\btheta_0$.
\begin{align}
    \mathbb{E}\left[\btheta_0\right]
    =
    \sum_{\bi\in S} \bi P\left(\bi\right)
    .
\end{align}
By the law of total expectation \eqref{eq:law-total-expectation} and the martingale property \eqref{eq:martingale-property}, we then have
\begin{align}
    \mathbb{E}\left[\btheta_k\right]
    &=
    \sum_{\bi\in S} \mathbb{E}\left[\btheta_k|\btheta_{k-1}=\bi\right]P\left(\bi\right)
    \nonumber\\
    &=
    \sum_{\bi\in S} \bi P\left(\bi\right)
    \nonumber\\
    &=
    \mathbb{E}\left[\btheta_0\right]
    .
\end{align}
To reiterate, the expectation of every step of a martingale is the same 
\begin{align}
    \label{eq:martingale-unchanging-expectation}
    \mathbb{E}\left[\btheta_k\right]
    =
    \mathbb{E}\left[\btheta_0\right]
    .
\end{align}
The prorty \eqref{eq:martingale-unchanging-expectation} of martingales reflects the notion of a ``fair game''.

We can interpret each step of the martingale as playing one round of a game, with the different states reflecting your earnings.
Your expected future earnings at each round are equal to your initial earnings--thus ``on average'' you do not earn or lose any money. 
This property holds for the namesake of martingale: the martingale betting strategy.
In this strategy (which arose in 18$^{th}$ century France), you bet double or nothing for every round of some win/lose game until you win, thus recovering all the previous money you had lost. 
the problem with this strategy is that in real life you may run out of money before you can double your bet.
The generalization of the probability of this happening--that is the probability of going bankrupt before winning a game--is called the ``gamblers ruin'' problem.

%=================================================================================
\section{Martingale $\neq$ Markov chain}

The martingale property \eqref{eq:martingale-property} looks superficially similar to the definition of a Markov process, but it's not.
A Markov process does not need to be a martingale process, and a martingale process does not need to be a Markov process.
As with Chptr.~\ref{chap:markov-chains}, we only consider discrete martingales--that is martingales where there are discrete jumps in time between states.

To see how a martingale and Markov chain are different, consider a Markov chain $\{\btheta_1,...\}$, with transition probabilities $P\left(\bj|\bi\right)$.
In order for the Markov chain to be a martingale, it must satisfy
\begin{align}
    \sum_{\bj\in S}  \bj P\left(\btheta_{k+1}=\bj|\btheta_{k}=\bi\right) = \bi
    .
\end{align}
Clearly this property is not always satisfied for all Markov chains (for example, consider a biased coin, with absorbing states at heads and tails--only for an unbiased coin will the outcome of the flips be a matingale).
We conclude that not every Markov chain is a martingale.

One can also show that not every martingale is a Markov chain.

%=================================================================================
\section{Stopping rules and Wald's equality}

A \textbf{stopping time} for a stochastic process $\btheta_i$ 

%=================================================================================
\section{Symmetric random walk\label{sec:symmetric-random-walk}}

A widely studied martingale is the \textbf{symmetric random walk}.
A symmetric random walk is a stochastic process, where at each step we have the random variable
\begin{align}
    \label{eq:def-random-walk-vector-sum}
    \bS_k
    \equiv
    \sum_{i=1}^k\btheta_i
    .
\end{align}
The $\btheta_i$ are identically distributed random variables $\{\btheta_i\}$ in an $n$-dimensional Euclidean space.
These variables have unit length, and their PDF to point in angular direction $\bphi=\left(\phi_0,...,\phi_{n-1}\right)$ is uniform.
The expectation for step $\btheta_k$ is then
\begin{align}
    \mathbb{E}\left[\btheta_{i}\right]
    =
    \int d\Omega_{n-1} \bu\left(\bphi\right)
    =
    0
    ,
\end{align}
where $\bu\left(\bphi\right)$ is a unit vector pointing in the direction $\bphi$.
We additionally have
\begin{align}
    \mathbb{E}\left[\left|\btheta_i\right|^2\right]
    =
    \int d\Omega_{n-1} \left|\bu\left(\bphi\right)\right|^2
    =
    \int d\Omega_{n-1} 
    =
    1
    .
\end{align}
Finally, as the vectors are independent, we have
\begin{align}
    \mathbb{E}\left[\btheta_i\cdot\btheta_j\right]
    =
    \mathbb{E}\left[\btheta_i\right]
    \cdot
    \mathbb{E}\left[\btheta_j\right]
    =
    \bZero
    .
\end{align}

These facts imply that
\begin{align}
    \mathbb{E}\left[\bS_k\right]
    =&
    \bZero
    ,\\
    \mathbb{E}\left[\left|\bS_k\right|^2\right]
    =&
    k
    .
\end{align}
Notice that by the convexity of the expectation, we have $\mathbb{E}\left[\left|\bS_k\right|\right]\leq \sqrt{k}$.

The sequence $\{\bS_k\}$ defines a martingale, as
\begin{align}
    \mathbb{E}\left[\bS_{k+1}|\bS_{k}=\bs,\bS_{k-1}=\bs_{k-1},...,\bS_1=\bs_1\right]
    =&
    \int d\Omega_{n-1} \left(\bs + \bu\left(\bphi\right)\right)
    \nonumber\\
    =&
    \bs
    .
\end{align}
The sequence $\{\left|\bS_{k}\right|^2 - k\}$ also defines a martingale.
We define the random variable $\bY_k \equiv \bS_k - k$ to reduce clutter.
We define $\by \equiv \bs - k$, and calculate:
\begin{align}
    \mathbb{E}\left[\bY_{k+1}|\bY_{k}=\by,\bY_{k-1}=\by_{k-1},...,\bY_1=\by_1\right]
    =&
    \int d\Omega_{n-1} \left(
        \left|\bs + \bu\right|^2
        -
        \left(k+1\right)
    \right)
    \nonumber\\
    =&
    \left|\bs\right|^2
    +
    1
    -
    \left(k+1\right)
    \nonumber\\
    =&
    \left|\bs\right|^2
    -
    k
    \nonumber\\
    =&
    \by
    .
\end{align}

