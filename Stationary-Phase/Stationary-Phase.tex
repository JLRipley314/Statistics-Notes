%===============================================================================
\section{Stationary phase approximation}

Here we review the \textbf{stationary phase approximation} for the Fourier transform. 
For more discussion see \cite{bender1999advanced}.
Consider a complex function, which we write as
\begin{align}
    B\left(t\right) 
    =
    A\left(t\right)e^{i\phi\left(t\right)}
    .
\end{align}
The Fourier transform is
\begin{align}
    \label{eq:FT-B}
    \tilde{B}\left(f\right)
    =
    \int_{-\infty}^{\infty}dt A\left(t\right) e^{i\phi\left(t\right) - 2\pi i ft}
    .
\end{align}
We imagine $A\left(t\right)$ is a slowly varying function, while $\phi\left(t\right)$
is rapidly varying. We then expect that the integral for $\tilde{B}\left(f\right)$
will be dominated by the stationary points of $\phi\left(t\right) - 2\pi ft$,
that is the points where
\begin{align}
\label{eq:stationary-points-phi}
    \frac{d\phi}{dt}
    -
    2\pi f
    =
    0
    .
\end{align}
This can be more formally justified by the Riemann-Lebesgue lemma, 
which states that 
\begin{align}
    \lim_{x\to\infty}\int_{a}^{b} dt e^{ixt}A\left(t\right)
    =
    0
    ,
\end{align}
provided $\int_a^bdt A\left(t\right)$ exists. 
We can extend $a,b\to\pm\infty$ so long as $A\left(t\right)$ is integrable.
Going back to \eqref{eq:FT-B}, we assume that $\phi\left(t\right) - 2\pi f t$
has one stationary point for each value of $f$, which we call $t_0\left(f\right)$.
That is, $t_0\left(f\right)$ is defined to solve the stationary phase equation 
\begin{align}
    \frac{d\phi}{dt}\Big|_{t=t_0}
    -
    2\pi f
    =
    0
    .
\end{align}
We Taylor series expand about the stationary point to quadratic order in $\phi$,
\begin{align}
    \phi\left(t\right)
    -
    2\pi ft
    =
    \phi\left(t_0\right)
    -
    2\pi ft_0
    +
    \frac{1}{2}\frac{d^2\phi}{dt^2}\Big|_{t=t_0}\left(t-t_0\right)^2
    +
    \mathcal{O}\left[\left(t-t_0\right)^3\right] 
    ,
\end{align}
insert this into \eqref{eq:FT-B}, and obtain
\begin{align}
    \tilde{B}\left(f\right)
    \approx&
    A\left(t_0\right) e^{i\phi\left(t_0\right) - 2\pi i f t_0}
    \int_{-\infty}^{\infty} dt 
    \exp\left[
        i \frac{1}{2}\frac{d^2\phi}{dt^2}\Big|_{t=t_0}\left(t-t_0\right)^2
    \right]
    \nonumber\\
    =&
    \left[\frac{1}{2}\frac{d^2\phi}{dt^2}\Big|_{t=t_0}\right]^{-1/2}
    A\left(t_0\right) e^{i\phi\left(t_0\right) - 2\pi i f t_0 + i\pi/4}
    \int_{-\infty}^{\infty} dx e^{-x^2} 
    \nonumber\\
    =&
    \left[\frac{1}{2\pi}\frac{d^2\phi}{dt^2}\Big|_{t=t_0}\right]^{-1/2}
    A\left(t_0\right) e^{i\phi\left(t_0\right) - 2\pi i f t_0 + i\pi/4}
    .
\end{align}
Using the chain rule, we can write $\phi\left(t_0\right)$ and $t_0$ 
as integral equations in terms of the frequency. We have
\begin{subequations}
\begin{align}
    t_0\left(f\right)
    =&
    \int^f df^{\prime}\frac{dt}{df}
    ,\\
    \phi\left(t_0\right)
    =&
    2\pi \int^f df^{\prime} \frac{dt}{df} f^{\prime}  
    .
\end{align}
\end{subequations}
Defining $\dot{f}\equiv df/dt$, we see that we can write the phase of
$\tilde{B}\left(f\right)$ as
\begin{align}
    \Psi
    \equiv&
    \phi\left(t_0\right)
    -
    2\pi f t_0 
    +
    \frac{\pi}{4}
    \nonumber\\
    =&
    2\pi \int^f df^{\prime} \frac{1}{\dot{f}}\left( f^{\prime} - f \right)
    +
    \frac{\pi}{4}
    .
\end{align}
