%===============================================================================
\section{Brief review of complex analysis}

For a function $A\left(t\right)$ that is singular at infinity, 
the \textbf{Cauchy principal value} is defined to be
\begin{align}
    \mathrm{p.v.}\int_{-\infty}^{\infty} dt A\left(t\right)
    =
    \lim_{T\to\infty}\int_{-T}^{T}dt A\left(t\right)
    .
\end{align}
For complex-valued functions $A\left(z\right)$ that are singular at a point $z_0$, 
the Cachy principal value is defined to be the limit of the deformation of the integral 
$C$ by a disk of radius $\epsilon$ centered around $z_0$
\begin{align}
    \mathrm{p.v.}\int_{C} dz A\left(z\right)
    =
    \lim_{\epsilon\to0^+}
        \int_{C\left(\epsilon\right)}dz A\left(z\right)
    .
\end{align}
This can also be written as
\begin{align}
    \mathrm{p.v.}\int_{C} dz A\left(z\right)
    =
    \lim_{\epsilon\to0^+}
        \left(
            \int_{-\infty}^{z_0-\epsilon}dz A\left(z\right)
            +
            \int_{z_0+\epsilon}^{\infty}dz A\left(z\right)
        \right)
    .
\end{align}
%===============================================================================
\section{The Fourier transform}

We briefly review Fourier transforms, along with a helpful transforms that
are used in signal processing.

The one-dimensional \textbf{Fourier transform} and its inverse are
\begin{subequations}
\begin{align}
    A\left(t\right)
    =
    \mathcal{F}^{-1}\left[\tilde{A}\left(f\right)\right]\left(t\right)
    =&
    \int_{-\infty}^{\infty}df e^{ 2\pi i ft} \tilde{A}\left(f\right)
    ,\\
    \tilde{A}\left(f\right)
    =
    \mathcal{F}\left[A\left(t\right)\right]\left(f\right)
    =&
    \int_{-\infty}^{\infty}df e^{-2\pi i ft} A\left(t\right)
    . 
\end{align}
\end{subequations}

The Fourier representation of the Dirac delta function $\delta\left(t\right)$ is
\begin{align}
    \tilde{\delta}\left(f\right)
    =
    \int_{-\infty}^{\infty}df e^{- 2\pi i f t} \delta\left(t\right)
    =
    1
    .
\end{align}

The \textbf{convolution} of two functions $A(t)$ and $B(t)$ are
\begin{align}
    \left(A* B\right)\left(t\right)
    \equiv
    \int_{-\infty}^{\infty}d\tau A\left(\tau\right)B\left(t-\tau\right)
    = 
    \int_{-\infty}^{\infty}d\tau A\left(t-\tau\right)B\left(\tau\right)
    .
\end{align}
The Fourier transform of the convlution is
\begin{align}
    \mathcal{F}\left[\left(A* B\right)\left(t\right)\right]\left(f\right)
    =&
    \int_{-\infty}^{\infty}d\tau \int_{-\infty}^{\infty}df \int_{-\infty}^{\infty}df^{\prime}
        e^{2\pi i f\tau} e^{2\pi i f^{\prime}\left(t-\tau\right)}
        \tilde{A}\left(f\right)\tilde{B}\left(f^{\prime}\right)
    \nonumber\\
    =&
    \int_{-\infty}^{\infty}df e^{2\pi i ft}
        \tilde{A}\left(f\right)\tilde{B}\left(f\right)
    .
\end{align}
That is, convolution in real space is multiplication in frequency space.

%===============================================================================
\section{The Laplace transform}

The \textbf{Laplace transform} of a function $A(t)$ is
\begin{subequations}
\begin{align}
    A\left(t\right)
    =
    \mathcal{L}^{-1}\left[A\left(\lambda\right)\right]\left(t\right)
    &=
    \frac{1}{2\pi i}\lim_{T\to\infty}
    \int_{\gamma-iT}^{\gamma+iT}d\lambda e^{\lambda t}\tilde{A}\left(\lambda\right)
    ,\\
    \tilde{A}\left(t\right)
    =
    \mathcal{L}\left[A\left(t\right)\right]\left(\lambda\right)
    &=
    \int_0^{\infty}dte^{-\lambda t}A\left(t\right)
    ,
\end{align}
\end{subequations}
Here $\gamma$ is a real number so that the contour path of integration is in
the region of convergence of $\tilde{A}\left(\lambda\right)$.
In effect, the inverse Laplace transform is like the inverse Fourier transform.
%===============================================================================
\section{The Hilbert transform}

The \textbf{Hilbert transform} of a function $A\left(t\right)$ is
\begin{subequations}
\begin{align}
    A\left(t\right)
    =
    \mathcal{H}^{-1}\left[\tilde{A}\left(\tau\right)\right]\left(t\right)
    &
    =
    -
    \frac{1}{\pi}\mathrm{p.v.}\int_{-\infty}^{\infty}d\tau\frac{\tilde{A}\left(\tau\right)}{t-\tau}
    ,\\
    \tilde{A}\left(\tau\right)
    =
    \mathcal{H}\left[A\left(t\right)\right]\left(\tau\right)
    &
    =
    \frac{1}{\pi}\mathrm{p.v.}\int_{-\infty}^{\infty}dt\frac{A\left(t\right)}{\tau-t}
    .
\end{align}
\end{subequations}
